\documentclass[12pt, a4paper]{article}

% ============================
% 1. STYLE & LANGUAGE
% ============================
% Load your new "Hausu" style first. 
% It handles: geometry, hyperref, titlesec, tikz, xcolor, fancyhdr, etc.
\usepackage{style} 
\usepackage[spanish, es-tabla, es-nodecimaldot]{babel}
\usepackage{csquotes}
\usepackage{pgfplots}
\pgfplotsset{compat=1.18}

% ============================
% 2. BIBLIOGRAPHY
% ============================
\usepackage[
backend=biber,
style=authoryear,
natbib=true,
maxcitenames=2
]{biblatex}
\addbibresource{references.bib}

% ============================
% 3. EXTRA DIAGRAM CONFIG
% ============================
% 'style.sty' loads TikZ core. Here we just load the 
% specific libraries you need for your diagrams in the chapters.
\usetikzlibrary{positioning, arrows.meta, fit, shapes.misc}

% ============================
% 4. METADATA
% ============================
% Note: I removed \textbf and \Huge because style.sty 
% now handles the font size and color automatically in the title page.
\title{Hausu (1977): Entre el Teatro Nō, la Televisualidad Pop y el Trauma Posbélico\\ \vspace{0.5cm} \large Un estudio cultural, histórico y computacional}

\author{Jorge Luis Mayorga Taborda}
\date{\today}

% ============================
% 5. DOCUMENT CONTENT
% ============================
\begin{document}
	
	% Generates the new "Hausu" graphical title page
	\maketitle 
	
	% The film strip header will automatically appear from page 2 onwards
	\newpage
	
	{
		\hypersetup{linkcolor=hausuDark} % Temporarily make TOC links black for readability
		\tableofcontents
	}
	\newpage
	
	% ============================
	% CHAPTERS
	% ============================
	
	% Tip: Use the new environments inside your chapters!
	% \begin{popanalysis}[Title] ... \end{popanalysis}
	% \begin{bloodnote}[Warning] ... \end{bloodnote}
	
	\section*{Resumen}

\noindent
\textit{Hausu} (Nobuhiko Obayashi, 1977) ha sido tradicionalmente descrita como un film de culto “inclasificable” debido a su estética saturada, su mezcla de horror y comedia y su montaje fragmentado. Este ensayo sostiene que la película no puede comprenderse únicamente desde las tendencias globales del proto-slasher ni desde el experimentalismo japonés de la época. Por el contrario, \textit{Hausu} emerge como el resultado singular de la convergencia de cuatro fuerzas culturales específicas del Japón de los años 1970: (1) la dramaturgia del fantasma femenino del teatro Nō; (2) la gramática visual hipersaturada de la televisión y la publicidad japonesa; (3) la lógica representacional del cuerpo femenino heredada del \textit{Pink Eiga} y el \textit{Roman Porno}; y (4) la memoria traumática pos-Hiroshima procesada mediante códigos pop-infantiles. A través de la comparación con obras europeas y estadounidenses contemporáneas, el ensayo identifica qué elementos son intercambiables y cuáles resultan irrepetibles fuera de Japón. La conclusión afirma que \textit{Hausu} constituye una singularidad estética que ilumina el cruce entre tradición, modernidad pop y trauma histórico en el Japón posbélico.





\begin{figure}[ht]
	\centering
	\begin{tikzpicture}[
		node distance=8mm,
		% Estilo de las cajas (Sin el comando title erróneo)
		box/.style={
			rectangle,
			rounded corners,
			draw=hausuDark,
			thick,
			fill=hausuGhost,
			align=left,
			font=\small\sffamily,
			inner sep=6pt,
			text width=0.75\linewidth,
			drop shadow={opacity=0.5, shadow xshift=2pt, shadow yshift=-2pt}
		},
		% Estilo del núcleo
		core/.style={
			rectangle,
			rounded corners,
			draw=hausuDark,
			ultra thick,
			fill=hausuBlood,
			text=white,
			align=center,
			font=\bfseries\large\sffamily,
			inner sep=8pt,
			text width=0.6\linewidth,
			drop shadow={opacity=0.8, shadow xshift=3pt, shadow yshift=-3pt}
		},
		% Flechas
		arrow/.style={-Latex, color=hausuOrange, line width=2pt}
		]
		
		% ---- Entradas japonesas ----
		% CORRECCIÓN: Quitamos "title=..." de aquí
		\node[box] (jp) {
			\textbf{\textcolor{hausuPurple}{Entradas japonesas}}\\[3pt]
			-- Teatro Nō: fantasma femenino, tiempo suspendido.\\
			-- TV y publicidad 70s: colores saturados, ritmo frenético.\\
			-- \textit{Pink Eiga}: cuerpo femenino fragmentado.\\
			-- Trauma pos-Hiroshima: memoria bélica y fantasía.
		};
		
		% ---- Entradas globales ----
		\node[box, below=of jp] (gl) {
			\textbf{\textcolor{hausuPurple}{Entradas globales}}\\[3pt]
			-- Horror juvenil / proto-slasher.\\
			-- Surrealismo sensorial europeo.\\
			-- Cine experimental 70s: anti-realismo.\\
			-- Efectos ópticos y trucos psicodélicos.
		};
		
		% ---- Nodo central: Hausu ----
		\node[core, below=of gl] (hausu) {
			\faCat \space \textit{HAUSU} (1977) \space \faGhost \\[3pt]
			\small Convergencia de Japón + Global
		};
		
		% ---- Salida / resultado estético ----
		\node[box, below=of hausu] (out) {
			\textbf{\textcolor{hausuPurple}{Salida: estética y legado}}\\[3pt]
			-- Horror infantil-pop y \textit{kawaii} siniestro.\\
			-- Lógica de ``videojuego'' (niveles, roles).\\
			-- Influencia en horror pop y anime posterior.
		};
		
		% Flechas
		\draw[arrow] (jp.south) -- (gl.north);
		\draw[arrow] (gl.south) -- (hausu.north);
		\draw[arrow] (hausu.south) -- (out.north);
		
	\end{tikzpicture}
	\caption{Esquema de la singularidad estética de \textit{Hausu}.}
\end{figure}
	\section{Introducción}

\textit{Hausu} (1977) ocupa una posición paradójica: es simultáneamente una obra profundamente japonesa y, a la vez, alineada con tendencias internacionales del horror juvenil de los años setenta. Su reputación como film “absurdo” o “inclasificable” suele ocultar la coherencia cultural profunda que sostiene su estética. Este ensayo propone que la rareza de \textit{Hausu} no deriva de su excentricidad formal, sino de la convergencia única de sistemas culturales japoneses que no coincidían en ninguna otra cinematografía mundial.

La hipótesis central de este trabajo sostiene que:

\begin{quote}
\textbf{\textit{Hausu} es irrepetible fuera de Japón porque articula simultáneamente cuatro dimensiones culturales: la ontología fantasmal del teatro Nō, la estética televisiva pop de los años 70, la representación simbólica del cuerpo femenino heredada del \textit{Pink Eiga}, y la posmemoria traumática de Hiroshima procesada a través de códigos infantiles.}
\end{quote}

La metodología combina análisis histórico, comparaciones fílmicas, estudios culturales y evidencia estética cuantitativa.

	% =========================================================
\section{Contexto cinematográfico japonés (1965--1977)}
% =========================================================


\subsection{La emergencia de estéticas híbridas y cineastas periféricos}
% -- Obayashi, Terayama, Suzuki; lenguaje publicitario, manga, tokusatsu, teatro experimental

\subsection{Panorama comercial: franquicias, blockbusters y segmentación del público}
% -- Datos de taquilla, éxito de Tora-san, irrupción del anime, influencia de Jaws y Star Wars

\subsection{El lugar de \textit{Hausu} en el sistema industrial japonés}
% -- Anomalía formal pero producto lógico de la diversificación, inserción industrial, target juvenil

\subsection{Conclusión: un cine entre la disolución del clasicismo y la irrupción de lo pop}
% -- Síntesis de las tensiones industriales, estéticas y culturales en el Japón de la poscrisis





El periodo comprendido entre mediados de los años sesenta y finales de los setenta constituye uno de los momentos más convulsos y transformadores de la historia del cine japonés. La erosión del sistema de estudios, el ascenso de nuevas sensibilidades juveniles, la competencia estructural de la televisión y la emergencia de circuitos alternativos —desde el \textit{pinku eiga} hasta el cine independiente— dieron lugar a un ecosistema fílmico heterogéneo, contradictorio y excepcionalmente permeable a propuestas radicales. En este contexto se inscribe \textit{Hausu} (Obayashi, 1977), cuya estética híbrida y su estructura fragmentaria son inseparables de la reconfiguración industrial y cultural del Japón cinematográfico tardomoderno.

% ---------------------------------------------------------
\subsection{Crisis estructural del sistema de estudios}
% ---------------------------------------------------------

Entre 1960 y 1975, el sistema de estudios japonés atravesó una de las crisis más severas de su historia, resultado de una convergencia de factores industriales, tecnológicos, sociales y culturales que desmantelaron el modelo vertical de producción, distribución y exhibición que había sostenido al cine japonés desde la posguerra. La magnitud del colapso queda evidenciada en las estadísticas de la Motion Picture Producers Association of Japan (Eiren): las admisiones anuales descendieron de 1.130 millones en 1960 a solo 356 millones en 1970, una pérdida de más del 68\% en apenas una década\footnote{Eiren: Motion Picture Producers Association of Japan, \textit{Historical Statistics of Film Attendance and Screens} (1960–1975).}. A mediados de los setenta, la caída acumulada superaba el 80\%, acompañada por el cierre masivo de salas, la reducción del número de estrenos y la reestructuración de los cinco grandes estudios —Toho, Shochiku, Nikkatsu, Toei y Daiei—.

Wada-Marciano resume esta transformación como el paso de un sistema coherente a una constelación fragmentada de unidades productivas semi-independientes, señalando que los estudios \enquote{apenas lograron sobrevivir durante una prolongada decadencia económica que comenzó en los años 60} \citep[37]{WadaMarciano2008}. Esta descentralización forzada trajo consigo un debilitamiento del sistema de estrellas, una reconfiguración del rol del productor y una creciente dependencia de productos de bajo coste.

Las cifras de producción ilustran este proceso: de los 548 largometrajes lanzados en 1960, se pasó a 305 en 1969 y a apenas 228 en 1975, según Anderson y Richie \citep{AndersonRichie1982}. La desaparición del control vertical obligó a adoptar estrategias defensivas: externalización de rodajes, diversificación temática, incremento de contratos por proyecto y desarticulación progresiva del modelo de repertorio interno. Shochiku quedó debilitado tras la muerte de Kido Shiro; Nikkatsu abandonó la producción generalista en 1971 para especializarse en el cine erótico con la línea \textit{Roman Porno}; Daiei se declaró en bancarrota ese mismo año, siendo absorbida en parte por Tokuma Shoten; mientras Toei desplazaba su estrategia hacia el \textit{tokusatsu} televisivo y la animación juvenil.

En este nuevo escenario, el cine japonés dejó de funcionar como un sistema cohesionado para convertirse en un ecosistema de supervivencia, donde lo que Zahlten llama “modos de producción periféricos” —incluyendo el \textit{pinku eiga}, la televisión, el cine de autor autofinanciado y la publicidad— comenzaron a ocupar un papel central en la creación de nuevas formas fílmicas \citep{Zahlten2017}. Como señala Rayns, la presión financiera obligó a los estudios a reducir sus presupuestos medios hasta niveles inéditos, lo cual favoreció la adopción de estéticas más económicas, la experimentación con géneros marginales y el abandono de las fórmulas narrativas tradicionales \citep{Rayns1999}.

Este contexto es indispensable para comprender cómo un film como \textit{Hausu}, con su estética fragmentaria, su bajo presupuesto y su tono híbrido entre publicidad, comedia escolar y horror fantástico, pudo ser producido por un estudio como Toho: no como excepción absoluta, sino como síntoma de una industria dislocada que había comenzado a buscar en lo excéntrico, lo juvenil y lo mediáticamente híbrido una posible vía de supervivencia.


\begin{figure}[htbp]
  \centering
  \begin{tikzpicture}

    % ---------- EJE IZQUIERDO: ENTRADAS + SALAS ----------
    \begin{axis}[
      width=0.95\textwidth,
      height=0.60\textwidth,
      xmin=1955, xmax=1990,
      ymin=0, ymax=1200,
      axis y line*=left,
      axis x line*=bottom,
      xlabel={Año},
      ylabel={Entradas (millones) / Salas ($\times 10$)},
      xtick={1955,1960,1965,1970,1975,1980,1985,1990},
      ymajorgrids=true,
      grid style={dashed, hausuDark!20}, 
      ticklabel style={font=\small\sffamily},
      label style={font=\small\sffamily\bfseries},
      legend style={
        at={(0.02,0.97)},
        anchor=north west,
        draw=hausuDark,
        fill=white,
        font=\small\sffamily,
        drop shadow
      },
    ]

      % --- Serie 1: Entradas totales (ROJO SANGRE) ---
      \addplot[
        very thick,
        hausuBlood,
        mark=*,
        mark options={hausuBlood,scale=1.2}
      ]
      coordinates {
        (1955,923.6) (1960,1087.2) (1965,448.2) (1970,337.3)
        (1975,304.8) (1980,330.3) (1985,328.6) (1990,317.9)
      };
      \addlegendentry{Entradas (millones)}

      % --- Serie 2: Salas de cine (TURQUESA/AZUL) ---
      \addplot[
        very thick,
        hausuTeal,
        dashed,
        mark=square*,
        mark options={hausuTeal,scale=1.0}
      ]
      coordinates {
        (1955,518.4) (1960,745.7) (1965,464.9) (1970,324.6)
        (1975,244.3) (1980,236.4) (1985,213.7) (1990,183.6)
      };
      \addlegendentry{Salas de cine ($\times 10$)}

    \end{axis}

    % ---------- EJE DERECHO: HOGARES CON TV (%) ----------
    \begin{axis}[
      width=0.95\textwidth,
      height=0.60\textwidth,
      xmin=1955, xmax=1990,
      ymin=0, ymax=100,
      axis y line*=right,
      axis x line=none,
      ylabel={Hogares con TV (\%)},
      ytick={0,20,40,60,80,100},
      yticklabel style={font=\small\sffamily},
      label style={font=\small\sffamily\bfseries},
      legend style={
        at={(0.98,0.03)},
        anchor=south east,
        draw=hausuDark,
        fill=white,
        font=\small\sffamily,
        drop shadow
      },
    ]

      % Datos aproximados en % (coherentes con la literatura)
      \addplot[
        very thick,
        hausuGreen,
        dotted,
        mark=triangle*,
        mark options={hausuGreen,scale=1.2, draw=black}
      ]
      coordinates {
        (1955,  1)  % TV incipiente
        (1960, 55)  % ~mitad de hogares
        (1965, 90)  % difusión casi completa en urbano
        (1970, 94)  % >90% hogares
        (1975, 98)
        (1980, 99)
        (1985, 99)
        (1990, 99)
      };
      \addlegendentry{Hogares con TV (\%)}

    \end{axis}

  \end{tikzpicture}

  \caption{Colapso de la audiencia cinematográfica y auge de la televisión en Japón (1955--1990).}
  \label{fig:asistencia-salas-tv-japon1955-1990}
\end{figure}

Los datos de asistencia y número de salas para el periodo 1955–1990 se han reconstruido a partir de las series oficiales de la Motion Picture Producers Association of Japan (EIREN), sintetizadas en \citep{EirenStats, Coates2020}.

Los valores aproximados de penetración televisiva (\%) representan una curva estilizada
a partir de diversos estudios sobre la difusión de la televisión en Japón, que sitúan
la presencia de aparatos en torno al 50--55\% de los hogares en 1960, por encima del
90\% en 1970 y cerca de la saturación (98--99\%) a finales de los años setenta y
ochenta \citep{Andrews2016, Yoshimi2010, StandardLivingJapan, TVPenetrationGraph}.


% ---------------------------------------------------------
\subsection{Competencia de la televisión y reconfiguración del consumo audiovisual}
% ---------------------------------------------------------

Uno de los factores más decisivos en la transformación del ecosistema cinematográfico japonés entre 1960 y 1977 fue la expansión masiva de la televisión como medio doméstico dominante. En apenas quince años, el televisor pasó de ser un objeto de lujo urbano a un electrodoméstico universal: en 1960, poco más del 55\% de los hogares japoneses poseían un televisor; para 1975, la cifra superaba el 95\%, consolidando a la televisión como el medio central en la vida cotidiana del país\footnote{Ministerio de Asuntos Internos y Comunicaciones de Japón (MIC), \textit{White Paper on Information and Communications}, 1975.}. 

Esta transformación tuvo un impacto inmediato sobre el cine, que perdió tanto cuota de pantalla como centralidad simbólica. La asistencia a salas cayó en paralelo a la expansión televisiva, marcando una doble redistribución del consumo audiovisual: primero, en términos cuantitativos (menos espectadores en cines); segundo, en términos cualitativos (nuevos hábitos, formatos y ritmos visuales). Las películas dejaron de ser el espacio privilegiado para el espectáculo visual, y debieron competir con un medio que ofrecía contenido gratuito, accesible y continuo. Como señalan \citeauthor{Ivy1995} y \citeauthor{WadaMarciano2008}, esta competencia no solo afectó a los ingresos de taquilla, sino que transformó el horizonte perceptivo de los espectadores, habituados ya a una lógica visual basada en el montaje rápido, la estética serializada y la apelación inmediata\citep{Ivy1995,WadaMarciano2008}.

A nivel industrial, los estudios intentaron responder con múltiples estrategias: adopción de técnicas televisivas, promoción de franquicias familiares, y contratación de directores formados en la publicidad televisiva. Nobuhiko Obayashi es un caso paradigmático: tras firmar más de 2.000 anuncios televisivos en los años 60 y 70, desarrolló una sensibilidad visual marcada por el ritmo hiperfragmentado, la saturación cromática, la sobreimpresión de texto y la teatralidad artificial. En \textit{Hausu}, estos recursos aparecen transpuestos al largometraje con una libertad compositiva inusual, evidenciando el tránsito de códigos televisivos al cine narrativo.

La figura \ref{fig:cinema_vs_tvs} presenta la relación inversa entre la caída de espectadores en salas y el crecimiento de la posesión de televisores por hogar, ilustrando la correlación directa entre ambos fenómenos.

\begin{figure}[t]
	\centering
	\begin{tikzpicture}
		\begin{axis}[
			width=\linewidth,
			height=6cm,
			xmin=1960, xmax=1975,
			ymin=0, ymax=120,
			axis y line*=left,
			ylabel={\% de hogares con TV},
			xlabel={Año},
			xtick={1960,1965,1970,1975},
			ytick={0,20,40,60,80,100},
			grid=both,
			legend style={at={(0.03,0.97)},anchor=north west,font=\scriptsize},
			]
			\addplot+[mark=square*, blue] coordinates {
				(1960,55)
				(1965,78)
				(1970,91)
				(1975,96)
			};
			\addlegendentry{Penetración TV}
		\end{axis}
		\begin{axis}[
			width=\linewidth,
			height=6cm,
			xmin=1960, xmax=1975,
			ymin=0, ymax=1200,
			axis y line*=right,
			axis x line=none,
			ylabel={Millones de admisiones al cine},
			ytick={0,200,400,600,800,1000,1200},
			legend style={at={(0.97,0.03)},anchor=south east,font=\scriptsize},
			]
			\addplot+[mark=*, red] coordinates {
				(1960,1130)
				(1965,702)
				(1970,356)
				(1975,244)
			};
			\addlegendentry{Admisiones cine}
		\end{axis}
	\end{tikzpicture}
	\caption{Crecimiento de la televisión y caída de asistencia al cine en Japón (1960–1975).}
	\label{fig:cinema_vs_tvs}
\end{figure}

Esta transformación no solo redirigió el flujo de espectadores, sino que forzó una reconversión estética: el cine japonés de los setenta, especialmente en sus expresiones más radicales o experimentales, asumió las condiciones del lenguaje televisivo como punto de partida. \textit{Hausu}, en este sentido, representa una forma de reciclaje postindustrial de los restos del cine clásico: una película que habita la pantalla grande con los códigos visuales de la pantalla chica, anticipando así la convergencia medial que dominaría las décadas posteriores.


% ---------------------------------------------------------
\begin{table}[t]
	\centering
	\caption{Comparativa entre asistencia al cine, hogares con televisión y número de salas en Japón (1960–1975)}
	\label{tab:tv_vs_cine_japan}
	\begin{tabular}{cccc}
		\toprule
		Año & Admisiones al cine (millones) & Penetración TV (\%) & Número de salas \\
		\midrule
		1960 & 1,130 & 55 & 7,457 \\
		1965 & 702 & 86 & 4,945 \\
		1970 & 356 & 95 & 3,602 \\
		1975 & 230 & 98 & 2,547 \\
		\bottomrule
	\end{tabular}
	\vspace{0.5em}
	\raggedright\footnotesize
	Fuentes: Eiren (MPPAJ); NHK Broadcasting Culture Research Institute; Sharp (2011); Desser (1988).
\end{table}

\begin{figure}[t]
	\centering
	\begin{tikzpicture}
		\begin{axis}[
			width=\linewidth,
			height=7cm,
			xlabel={Año},
			xtick=data,
			xticklabel style={rotate=45},
			ylabel={Valores (escala relativa)},
			legend style={at={(0.01,0.99)},anchor=north west,font=\small},
			grid=both,
			ymin=0,
			xtick={1960,1965,1970,1975}
			]
			\addplot+[mark=*] coordinates {(1960,1130) (1965,702) (1970,356) (1975,230)};
			\addlegendentry{Asistencia al cine (millones)}
			
			\addplot+[mark=square*] coordinates {(1960,55) (1965,86) (1970,95) (1975,98)};
			\addlegendentry{Penetración TV (\%)}
			
			\addplot+[mark=triangle*] coordinates {(1960,7457) (1965,4945) (1970,3602) (1975,2547)};
			\addlegendentry{Número de salas}
		\end{axis}
	\end{tikzpicture}
	\caption{Evolución comparada de asistencia al cine, penetración de TV y número de salas en Japón (1960–1975)}
	\label{fig:tv_vs_cine_plot}
\end{figure}



\subsection{Transformación de géneros: del jidaigeki al pinku eiga}
% -- Auge del cine erótico, cine de explotación, cine juvenil, pinky violence, Roman Porno





% ---------------------------------------------------------
\subsection{Producción barata, géneros híbridos y auge del cine erótico}
% ---------------------------------------------------------

La contracción del sistema clásico abrió un espacio para formas de producción más económicas, rápidas y estilísticamente flexibles. Entre estas destacan el \textit{pinku eiga} y, desde 1971, la línea \textit{Roman Porno} de Nikkatsu, documentadas en profundidad por \citeauthor{Weisser1998} \citep{Weisser1998}, \citeauthor{Sharp2011} \citep{Sharp2011} y \citeauthor{McDonald1992} \citep{McDonald1992}. \citeauthor{Zahlten2017} \citep{Zahlten2017} demuestra que, hacia finales de los setenta, casi el 80\% de los estrenos japoneses pertenecían a categorías de explotación erótica, acción o híbridos sensacionalistas.

Toei desarrolló en paralelo la serie de \textit{Pinky Violence}, que integraba delincuencia juvenil, erotismo estilizado, estética \textit{pop} y una marcada saturación cromática, como en \textit{Female Prisoner 701: Scorpion} (Itō, 1972), \textit{Sex and Fury} (Ishii, 1973) o \textit{Delinquent Girl Boss} (Ozawa, 1970). Este ciclo, como han señalado Sharp y Standish, funcionó como una respuesta económica al colapso del mercado y como un laboratorio formal para nuevas sensibilidades visuales \citep{Standish2011}.

La centralidad de la televisión en la vida cotidiana japonesa —analizada por \citeauthor{Ivy1995} \citep{Ivy1995} y \citeauthor{WadaMarciano2008} \citep{WadaMarciano2008}— también reconfiguró el lenguaje visual del cine. La mezcla de animación, trucajes ópticos, montaje frenético y estética \textit{tokusatsu} que caracteriza \textit{Hausu} es inseparable de este ecosistema mediático.

% ---------------------------------------------------------
\subsection{Un ecosistema híbrido: hacia la estética de \textit{Hausu}}
% ---------------------------------------------------------

El conjunto de estas transformaciones produjo lo que Desser denomina un \enquote{cinema of rupture}, caracterizado por hibridación estética, experimentación formal y descentralización industrial \citep{Desser1988}. Cineastas procedentes de la publicidad y de la televisión —entre ellos Nagisa Ōshima, Shuji Terayama y Nobuhiko Obayashi— encontraron un espacio para desarrollar estrategias narrativas y visuales no convencionales, apoyadas en presupuestos modestos y mayor libertad formal.

En este sentido, \textit{Hausu} debe entenderse menos como una excentricidad aislada y más como un producto directo de esta fase terminal del sistema clásico: un film financiado y distribuido por Toho, pero concebido desde la lógica fragmentaria y lúdica de la publicidad televisiva, el \textit{shōjo manga} y la experimentación óptica. Obayashi llega al largometraje después de haber firmado cientos de anuncios televisivos y de haber trabajado en el margen del cine experimental en 16 mm; su sensibilidad visual se forma, por tanto, en un régimen de imágenes pensado para ser visto en pantallas pequeñas, interrumpidas por cortes publicitarios y dirigidas a un público juvenil acostumbrado a la saturación cromática y al montaje agresivo. La presencia de protagonistas colegialas, la segmentación del relato en \enquote{episodios} casi autónomos (el ataque del piano, la escena del pozo, la habitación del tatami) y la superposición de capas gráficas —títulos, cortinillas, fondos pintados— traducen directamente al formato de largometraje los códigos del anuncio televisivo y del \textit{shōjo} seriado.

Como señalan \citeauthor{Richie1990} \citep{Richie1990} y \citeauthor{Rayns1999} \citep{Rayns1999}, es precisamente este periodo de reestructuración el que facilita la emergencia de cineastas \enquote{híbridos}, capaces de integrar recursos del lenguaje televisivo, la cultura juvenil y las artes visuales dentro de un marco aún industrial pero crecientemente descentrado. Autores como Ōshima, Terayama u Obayashi operan en un espacio intermedio entre el estudio y la producción independiente, aprovechando los resquicios de un sistema en crisis para introducir dispositivos formales considerados hasta entonces propios de la vanguardia o de la cultura popular \enquote{menor}. En \textit{Hausu}, esa hibridez se hace visible en la coexistencia de decorados y maquetas de estudio con transparencias toscas, animaciones superpuestas y efectos de montaje que recuerdan tanto al cine de vanguardia de los sesenta como a los programas de variedades y a los anuncios de refrescos.

Por ello, la estética saturada y el carácter híbrido de \textit{Hausu} no deberían interpretarse como anomalías puramente idiosincráticas, sino como síntomas de un sistema en proceso de descomposición y, simultáneamente, en estado de creatividad expansiva. La película cristaliza, en forma de comedia de horror juvenil, la transición de un régimen de producción basado en el control centralizado del estudio a otro atravesado por la televisión, el marketing y la segmentación generacional del público. Más que un \enquote{accidente} en la filmografía de Toho, \textit{Hausu} funciona como índice de hasta qué punto el cine japonés de la década de 1970 se había vuelto poroso a las formas, ritmos y fantasías producidos fuera de la sala oscura.

% ---------------------------------------------------------
\subsection{Éxitos comerciales, géneros dominantes y tendencias industriales (1965--1977)}
% ---------------------------------------------------------

En paralelo al colapso del sistema clásico, el mercado cinematográfico japonés produjo una serie de éxitos comerciales que revelan las tensiones entre tradición genérica, modernización cultural y estrategias de supervivencia industrial. A diferencia del modelo hollywoodense, donde los grandes éxitos de taquilla estaban dominados por superproducciones, el sistema japonés del periodo se articuló en torno a franquicias longevas, cine juvenil, \textit{tokusatsu}, cine erótico y películas familiares de bajo presupuesto.

En términos estrictamente industriales, las estadísticas de la Motion Picture Producers Association of Japan (Eiren) muestran una caída dramática en el número de espectadores: de 372,7 millones de entradas vendidas en 1965 se pasa a 254,8 millones en 1970 y a apenas 165,2 millones en 1977.\footnote{Datos de Eiren para los años 1965--1977: número de pantallas, películas estrenadas, admisiones y cuota de mercado de películas japonesas frente a importadas \citep{EirenStats}.} Al mismo tiempo, la cuota de ingresos de las películas japonesas desciende del 66{,}7\% al 44{,}4\% en 1975, antes de recuperarse ligeramente hasta el 50{,}8\% en 1977.\textsuperscript{\thefootnote} Esta combinación de pérdida de público y creciente competencia del cine importado generó un entorno de alto riesgo donde los estudios buscaron refugio tanto en los géneros de explotación como en franquicias de bajo coste y alto reconocimiento.

\begin{table}[t]
	\centering
	\caption{Indicadores básicos del mercado cinematográfico japonés (1965--1977)}
	\label{tab:eiren_basic_65_77}
	\begin{tabular}{lcccc}
		\toprule
		Año & Admisiones\textsuperscript{a} & Películas Jap. & Películas Imp. & Cuota Jap./Imp.\textsuperscript{b} \\
		\midrule
		1965 & 372{,}7 & 487 & 264 & 66{,}7\% / 33{,}3\% \\
		1970 & 254{,}8 & 423 & 236 & 59{,}4\% / 40{,}6\% \\
		1973 & 185{,}3 & 405 & 252 & 56{,}0\% / 44{,}0\% \\
		1975 & 174{,}0 & 333 & 225 & 44{,}4\% / 55{,}6\% \\
		1977 & 165{,}2 & 337 & 221 & 50{,}8\% / 49{,}2\% \\
		\bottomrule
	\end{tabular}
	
	\vspace{0.5em}
	\raggedright\footnotesize
	\textsuperscript{a} Admisiones en millones de espectadores (columna ``Number of Admissions'' de Eiren, expresada en miles).\par
	\textsuperscript{b} Cuota de ingresos de distribución de películas japonesas e importadas.
\end{table}

\begin{figure}[t]
	\centering
	\begin{tikzpicture}
		\begin{axis}[
			width=\linewidth,
			height=6cm,
			xmin=1965, xmax=1977,
			ymin=150, ymax=380,
			xlabel={Año},
			ylabel={Admisiones (millones)},
			xtick={1965,1970,1973,1975,1977},
			ytick={150,200,250,300,350},
			grid=both,
			legend style={at={(0.02,0.98)},anchor=north west,font=\scriptsize},
			]
			\addplot+[mark=*] coordinates {
				(1965,372.7)
				(1970,254.8)
				(1973,185.3)
				(1975,174.0)
				(1977,165.2)
			};
			\addlegendentry{Admisiones totales}
		\end{axis}
	\end{tikzpicture}
	\caption{Descenso de admisiones en el cine japonés (1965--1977) según datos de Eiren.}
	\label{fig:eiren_admissions_plot}
\end{figure}

Sobre este trasfondo de contracción del mercado se articulan distintos vectores genéricos que explican el lugar de \textit{Hausu} en 1977.

% ---------------------------------------------------------
\paragraph{1. El dominio del \textit{tokusatsu}: Godzilla, monstruos y catástrofes}
% ---------------------------------------------------------

Durante los años sesenta, la taquilla japonesa siguió liderada por los filmes de monstruos y ciencia ficción de Toho. Títulos como
\textit{Mothra vs. Godzilla} (Honda, 1964), \textit{Invasion of Astro-Monster} (Honda, 1965),
\textit{Ebirah, Horror of the Deep} (Honda, 1966), \textit{Son of Godzilla} (Honda, 1967) o
\textit{Destroy All Monsters} (Honda, 1968) mantuvieron una presencia constante entre los filmes nacionales más vistos del periodo, con varios de ellos situándose entre los líderes anuales de ingresos.\citep{RyfleGodzilla2000,GodzillaFranchise}

Los presupuestos de esta serie oscilaban, según estimaciones de Ryfle y Godziszewski, entre los 4 y los 5 millones de yen de la época (aproximadamente 300.000--500.000 dólares al cambio de entonces), cifras relativamente bajas para estándares internacionales, pero suficientes para sostener los efectos especiales del departamento de Eiji Tsuburaya.\citep{RyfleGodzilla2000} Toho equilibraba así dos líneas de supervivencia: por un lado, las franquicias de monstruos consolidadas; por otro, la experimentación con nuevos formatos destinados al público juvenil, en diálogo con la televisión y el \textit{merchandising}.

% ---------------------------------------------------------
\paragraph{2. El auge del cine juvenil y las \enquote{Nikkatsu Action Films}}
% ---------------------------------------------------------

A partir de mediados de los sesenta, Nikkatsu posicionó el cine juvenil de acción como un pilar comercial. Actores como Yūjirō Ishihara, Akira Kobayashi y Joe Shishido protagonizaron títulos hoy canónicos como
\textit{Tokyo Drifter} (Suzuki, 1966),
\textit{Branded to Kill} (Suzuki, 1967) o
\textit{A Colt Is My Passport} (Nomura, 1967).\citep{Sharp2011}
Aunque estos filmes desarrollaron un estilo visual altamente innovador —color plano, decorados \textit{pop}, montaje abrupto—, muchas de las \textit{Nikkatsu Action Films} tendían a rendimientos discretos, lo que aceleró la crisis interna del estudio y su giro radical hacia la línea \textit{Roman Porno} en 1971.\citep{Sharp2011,SecondYouthNikkatsu}

En otras palabras, el cine juvenil de acción funcionó como laboratorio estético más que como motor económico sostenido. El hecho de que obras hoy canonizadas como las de Seijun Suzuki fueran consideradas fracasos o productos problemáticos por la dirección de Nikkatsu ilustra la tensión entre experimentación formal y supervivencia financiera en este periodo.

% ---------------------------------------------------------
\paragraph{3. Boom del cine erótico comercial: \textit{Pinku Eiga} y \textit{Roman Porno}}
% ---------------------------------------------------------

Entre finales de los sesenta y finales de los setenta, el mayor volumen de estrenos y una parte sustantiva de los ingresos de Nikkatsu, Toei y numerosos estudios menores provino del cine erótico.\citep{Sharp2011,Weisser1998,Zahlten2017} Zahlten sitúa el \textit{pink film} como el verdadero motor de la transformación industrial: a partir de 1971, con el lanzamiento de la línea \textit{Roman Porno} de Nikkatsu y la adopción por parte de Toei del término \enquote{porno} en su promoción, el sector prácticamente se reconfigura alrededor de productos sexuales de bajo presupuesto y alta rotación.\citep{PinkFilm,Zahlten2017}

Según cálculos de Zahlten, hacia 1979 aproximadamente el 70--80\% de los estrenos en salas japonesas podían clasificarse como \textit{pink films}, \textit{Roman Porno} o alguna variante de \textit{sexploitation}.\citep{ZahltenScreen2019} Películas como
\textit{Apartment Wife: Affair in the Afternoon} (Tanaka, 1971),
\textit{Wife to Be Sacrificed} (Konuma, 1974) o
\textit{True Story of a Woman in Jail} (Ishii, 1975)
obtuvieron retornos muy superiores a su costo, con presupuestos que rara vez superaban los 100.000 dólares.\citep{Weisser1998}

Este boom revela un mercado dominado por géneros de alto rendimiento y bajo coste, lo que explica por qué Toho estuvo dispuesta a financiar un proyecto experimental como \textit{Hausu}: en un contexto donde el grueso de la producción ya se desplazaba hacia el cine erótico y de explotación, un film juvenil de terror cómico suponía un riesgo relativamente moderado frente a la caída general del mercado.

% ---------------------------------------------------------
\paragraph{4. Cine familiar y franquicias longevas: \textit{Tora-san}}
% ---------------------------------------------------------

El contrapunto fundamental al auge del cine juvenil y erótico fue el éxito masivo —y estable— de la serie \textit{Otoko wa Tsurai yo} (\textit{Tora-san}) de Shōchiku. Entre 1969 y 1997 se estrenaron 48 películas, muchas de las cuales figuran entre las diez más vistas de su año.\citep{AndersonRichie1982} La propia lista de filmes japoneses de mayor recaudación recoge varios títulos de la serie como líderes anuales: \textit{Tora-san's Love Call} (1971), \textit{Tora-san's Dream-Come-True} (1972), \textit{Tora-san's Lullaby} (1974) y \textit{Tora-san, the Intellectual} (1975) encabezan las tablas de taquilla nacional.\citep{HighestGrossingJPN}

\citeauthor{AndersonRichie1982} resumen el papel de la saga de forma tajante: 
\begin{displayquote}
	The Tora-san films had become Shochiku’s financial backbone throughout the 1970s.\citep[312]{AndersonRichie1982}
\end{displayquote}
Cada entrega se rodaba en pocas semanas, con presupuestos en torno a 150.000--200.000 dólares, y lograba beneficios constantes gracias a un público fiel que acudía no en busca de espectacularidad, sino de familiaridad afectiva y continuidad de personajes.

\begin{table}[t]
	\centering
	\caption{Líderes de taquilla japonesa por año (1965--1977)}
	\label{tab:top_jpn_65_77}
	\scriptsize
	\begin{tabularx}{\linewidth}{@{}lXlXX@{}}
		\toprule
		Año & Título (Japón) & Director & Género dominante & Observaciones \\
		\midrule
		1965 & \emph{Tokyo Olympiad} & Kon Ichikawa & Documental deportivo & Crónica oficial de los JJ.~OO. de Tokio. \\
		1966 & \emph{Ebirah, Horror of the Deep} & Ishirō Honda & \textit{Kaijū} / \textit{tokusatsu} & Entrada en la serie Godzilla. \\
		1967 & \emph{Japan's Longest Day} & Kihachi Okamoto & Drama bélico & Reconstrucción del final de la guerra. \\
		1968 & \emph{The Sands of Kurobe} & Kei Kumai & Drama de construcción & Épica industrial. \\
		1969 & \emph{Samurai Banners} & Hiroshi Inagaki & \emph{Jidaigeki} & Superproducción histórica. \\
		1970 & \emph{Men and War: Part I} & Satsuo Yamamoto & Drama bélico & Primera parte de trilogía. \\
		1971 & \emph{Tora-san's Love Call} & Yōji Yamada & Comedia familiar & Entrada en la saga \emph{Tora-san}. \\
		1972 & \emph{Tora-san's Dream-Come-True} & Yōji Yamada & Comedia familiar & Nueva entrega de \emph{Tora-san}. \\
		1973 & \emph{Submersion of Japan} & Shiro Moritani & Cine de catástrofes & Desastre geológico a escala nacional. \\
		1974 & \emph{Tora-san's Lullaby} & Yōji Yamada & Comedia familiar & \emph{Tora-san} consolidada como franquicia. \\
		1975 & \emph{Tora-san, the Intellectual} & Yōji Yamada & Comedia familiar & Continuidad de la fórmula Tora-san. \\
		1976 & \emph{The Human Revolution Continues} & Toshio Masuda & Drama histórico-religioso & Adaptación vinculada a Soka Gakkai. \\
		1977 & \emph{Mount Hakkoda} & Shirō Moritani & Drama bélico / desastre & Tragedia militar en la nieve. \\
		\bottomrule
	\end{tabularx}
\end{table}



La tabla \ref{tab:top_jpn_65_77} evidencia que, incluso en plena crisis, los mayores éxitos nacionales combinaban tres polos: \textit{tokusatsu} y catástrofes, dramas históricos/bélicos y comedias familiares seriadas. Frente a este paisaje, \textit{Hausu} aparece como un experimento juvenil-\textit{pop} relativamente excéntrico, pero insertado en una economía de franquicias y géneros muy marcada.

% ---------------------------------------------------------
\paragraph{5. Animación, \textit{anime} y blockbusters extranjeros}
% ---------------------------------------------------------

La consolidación del \textit{anime} televisivo (desde \textit{Astro Boy} en 1963) generó un flujo constante de largometrajes animados que, aunque no siempre dominaban las listas de taquilla, constituían un componente rentable y estratégico del mercado.\citep{Cavallaro2010} Títulos como
\textit{The Great Adventure of Horus, Prince of the Sun} (Takahata, 1968),
\textit{Panda! Go Panda!} (Takahata \& Miyazaki, 1972) o las primeras películas basadas en franquicias televisivas anticipan el modelo que explotará a gran escala a finales de los setenta con éxitos como \textit{Galaxy Express 999} (Rintarō, 1979).\citep{Galaxy999}

En paralelo, el mercado japonés se vio crecientemente dominado por \textit{blockbusters} estadounidenses, especialmente tras el impacto de \textit{Jaws} (Spielberg, 1975) y \textit{Star Wars} (Lucas, 1977), que demostraron el poder de las campañas de marketing globales también en el archipiélago.\citep{JawsJapan,StarWarsJapan} La siguiente tabla resume algunos de los principales éxitos de taquilla en el periodo inmediatamente anterior y posterior a \textit{Hausu}, combinando títulos domésticos y extranjeros:
\begin{table}[t]
	\centering
	\caption{Principales éxitos de taquilla en el mercado japonés (1975--1980)}
	\label{tab:japan_boxoffice_1975_1980}
	\scriptsize
	\begin{tabularx}{\linewidth}{@{}lXlXlcc@{}}
		\toprule
		Año & Título & Director & Género dominante & Origen & Presupuesto & Taquilla en Japón \\
		\midrule
		1975 & \emph{Tora-san, the Intellectual} & Yōji Yamada & Comedia popular / melodrama familiar & Japonesa & N/D & $\approx 1.19\times 10^{9}$ JPY \\
		1976 & \emph{The Human Revolution Continues} (\emph{Zoku Ningen Kakumei}) & Toshio Masuda & Drama histórico-religioso & Japonesa & N/D & $\approx 4.12\times 10^{9}$ JPY \\
		1977 & \emph{Mount Hakkoda} (\emph{Hakkōda-san}) & Shirō Moritani & Drama bélico / cine de desastre histórico & Japonesa & N/D & $\approx 2.59\times 10^{9}$ JPY \\
		1978 & \emph{Never Give Up} (\emph{Yasei no shōmei}) & Junya Satō & Thriller de acción / conspiración & Japonesa & N/D & $\approx 2.18\times 10^{9}$ JPY \\
		1979 & \emph{Galaxy Express 999} & Rintarō & Animación / ciencia ficción & Japonesa & N/D & $\approx 4.2\times 10^{9}$ JPY \\
		1980 & \emph{Kagemusha} & Akira Kurosawa & \emph{Jidaigeki} épico & Japonesa & $\approx 2.3\times 10^{9}$ JPY & $\approx 4.59\times 10^{9}$ JPY \\
		\midrule
		1975 & \emph{Jaws} & Steven Spielberg & Thriller de terror & Extranjera (EE.\,UU.) & $\approx 9\,\text{M USD}$ & $\approx 9.0\times 10^{9}$ JPY \\
		1977 & \emph{Star Wars} (\emph{Episode IV}) & George Lucas & \emph{Space opera} de ciencia ficción & Extranjera (EE.\,UU.) & $\approx 11\,\text{M USD}$ & $\approx 6.13\times 10^{9}$ JPY \\
		1977 & \emph{Close Encounters of the Third Kind} & Steven Spielberg & Ciencia ficción dramática & Extranjera (EE.\,UU.) & $\approx 20\,\text{M USD}$ & $\approx 6.6\times 10^{9}$ JPY \\
		1978 & \emph{Superman} & Richard Donner & Cine de superhéroes & Extranjera (RU/EE.\,UU.) & $\approx 55\,\text{M USD}$ & $\approx 5.6\times 10^{9}$ JPY \\
		\bottomrule
	\end{tabularx}
	
	\vspace{0.3em}
	{\footnotesize
		Notas: (1) Los presupuestos en USD corresponden al coste global de producción, no al gasto específico en Japón. (2) Las cifras de taquilla japonesa están sin ajustar por inflación y se basan en estimaciones de recaudación bruta o \emph{distributor rentals} convertidos a taquilla aproximada.\par}
\end{table}


Como muestra la tabla \ref{tab:japan_boxoffice_1975_1980}, el mercado japonés de la segunda mitad de los setenta estuvo dominado por una combinación de franquicias locales (\textit{Tora-san}), superproducciones históricas (\textit{Mount Hakkoda}, \textit{Kagemusha}), \textit{anime} de gran escala (\textit{Galaxy Express 999}) y \textit{blockbusters} hollywoodenses (\textit{Jaws}, \textit{Star Wars}, \textit{Close Encounters}, \textit{Superman}). En ese paisaje híbrido, \textit{Hausu} se sitúa como un experimento de estudio de presupuesto medio, formalmente radical pero inserto en una economía de géneros, franquicias y productos importados muy claramente estructurada.

En otras palabras, el público japonés de 1977 no buscaba necesariamente rupturas formales en su cine comercial: consumía dramas familiares, thrillers sobrios, filmes de catástrofes de gran presupuesto y, cada vez más, fenómenos globales de Hollywood. En ese paisaje, \textit{Hausu} —una película de terror cómico infantil-\textit{pop}, montada con lógica de anuncio y saturación psicodélica— era una anomalía estética, pero no una anomalía industrial. Como subrayan \citeauthor{Rayns1999} y \citeauthor{Zahlten2017}, en un mercado en contracción los estudios japoneses no podían competir frontalmente con Hollywood en términos de presupuesto; en cambio, tendían a abrazar lo experimental y lo híbrido para atraer a públicos nicho y diversificar riesgos.\citep{Rayns1999,Zahlten2017} En este contexto, \textit{Hausu} aparece como un síntoma particularmente visible de las tensiones del sistema: un film producido por un gran estudio, insertado en una economía dominada por el cine erótico, las franquicias familiares y los \textit{blockbusters} importados, que apuesta por un lenguaje visual radicalmente distinto para interpelar a un público juvenil fragmentado entre la sala de cine, la televisión y el \textit{anime}.

	% =========================================================
\section{Contexto cinematográfico global (1965--1977)}
% =========================================================

El periodo comprendido entre 1965 y 1977 representa una fase de transformación radical del ecosistema cinematográfico mundial. La irrupción de nuevas formas de consumo audiovisual, la consolidación de la televisión como medio dominante, la crisis de los estudios clásicos en Estados Unidos y Europa, el ascenso de movimientos modernistas y contraculturales y la proliferación de cines de explotación contribuyeron a reconfigurar el mapa estético e industrial del cine internacional. Durante estos años se erosionan las bases del clasicismo fílmico, se redefinen los géneros y emergen nuevas sensibilidades que privilegian la subjetividad, el cuerpo, la fragmentación y la experimentación formal \citep{NowellSmith1996, Cook2007}. 

% ---------------------------------------------------------
\subsection{Transformaciones industriales y tecnológicas}
% ---------------------------------------------------------

A mediados de los sesenta, la televisión había alcanzado una presencia masiva en los principales mercados occidentales, reduciendo la asistencia a salas, forzando el cierre de cines y obligando a los estudios a reconfigurar sus estrategias de producción. En Estados Unidos, la asistencia anual cayó de unos 90 millones de espectadores semanales en 1948 a menos de 17 millones en 1967 \citep{Gomery1992}. Europa siguió un patrón similar, con descensos significativos en Francia, Italia y Reino Unido; por ejemplo, la asistencia británica se redujo de aproximadamente 1{,}400 millones de entradas en 1950 a 176 millones en 1965 \citep{Harper2004}. 

La industria respondió con diversas estrategias: generalización del color, formatos panorámicos y de gran formato (CinemaScope, 70 mm), cine espectáculo y, en el otro extremo, un aumento considerable de producciones de bajo presupuesto. Los mercados empezaron a polarizarse entre grandes producciones de estudio (épicos, musicales, cine catástrofe) y nichos de explotación (horror, erotismo, ciencia ficción menor), fenómeno que se observa tanto en Hollywood como en Italia, España, Francia o Alemania Occidental \citep{Hutchings2004, Klinger2006}.

Esta contracción de la asistencia se puede visualizar mediante la caída conjunta de los principales mercados occidentales:

\begin{figure}[t]
	\centering
	\begin{tikzpicture}
		\begin{axis}[
			width=\linewidth,
			height=6cm,
			xmin=1950, xmax=1975,
			ymin=0, ymax=1500,
			xlabel={Año},
			ylabel={Entradas anuales (millones)},
			xtick={1950,1960,1965,1970,1975},
			ytick={0,200,400,600,800,1000,1200,1400},
			grid=both,
			legend style={at={(0.02,0.98)},anchor=north west,font=\scriptsize},
			]
			% Reino Unido (datos aproximados)
			\addplot+[mark=*] coordinates {
				(1950,1400)
				(1960,500)
				(1965,176)
				(1975,115)
			};
			\addlegendentry{Reino Unido}
			
			% Estados Unidos (entradas semanales transformadas a millones anuales aprox.)
			\addplot+[mark=square*] coordinates {
				(1950,3000)
				(1960,1500)
				(1967,900)
				(1975,800)
			};
			\addlegendentry{Estados Unidos}
		\end{axis}
	\end{tikzpicture}
	\caption{Descenso de la asistencia cinematográfica en EE.\,UU. y Reino Unido (1950--1975). Datos aproximados a partir de \cite{Gomery1992, Harper2004}.}
	\label{fig:global_admissions_plot}
\end{figure}

Aunque las cifras de la figura \ref{fig:global_admissions_plot} son aproximadas, ilustran una tendencia clara: el cine deja de ser el medio hegemónico de entretenimiento masivo y se ve obligado a repensar su modelo de negocio ante la competencia de la televisión y, hacia finales de los setenta, del vídeo doméstico emergente.

% ---------------------------------------------------------
\subsection{Movimientos modernistas y contraculturales}
% ---------------------------------------------------------

El período 1965--1977 coincide con el auge global del cine modernista, que redefinió el lenguaje fílmico mediante fragmentación narrativa, reflexividad, subjetividad extrema y rupturas espaciales y temporales. La Nouvelle Vague seguía produciendo obras clave —como \emph{Pierrot le fou} (Godard, 1965) o \emph{La Chinoise} (Godard, 1967)— que exploraban la relación entre política, juventud y medios de comunicación. En Italia, Antonioni continuó la investigación de la alienación moderna con \emph{Blow-Up} (1966) y \emph{Zabriskie Point} (1970). En Alemania, el Nuevo Cine Alemán consolidó figuras como Fassbinder (\emph{Liebe ist kälter als der Tod}, 1969), Herzog (\emph{Aguirre, der Zorn Gottes}, 1972) y Wenders (\emph{Der amerikanische Freund}, 1977) \citep{Elsaesser1989}.

Fuera de Europa, el modernismo cinematográfico adoptó formas específicas: el Cinema Novo brasileño de Glauber Rocha, el cine político latinoamericano, el cine de autor japonés de Ōshima y Yoshida, o el trabajo de cineastas africanos como Sembène. Todos estos movimientos compartían un interés por la desestabilización del realismo, la exploración de estados psicogeográficos y la crítica cultural; un horizonte que dialoga, aunque desde coordenadas distintas, con el anti-realismo y la imaginería pop-publicitaria que Obayashi desarrollaría en \textit{Hausu}.

% ---------------------------------------------------------
\subsection{Géneros populares y cines de explotación}
% ---------------------------------------------------------

Simultáneamente, la década vio una proliferación internacional de cines populares y de explotación, frecuentemente producidos con bajo presupuesto y dirigidos a públicos juveniles. Italia desarrolló el \emph{giallo} (Argento, \emph{L'uccello dalle piume di cristallo}, 1970), los \emph{poliziotteschi} (\emph{La polizia incrimina, la legge assolve}, Lenzi, 1973) y el horror gótico tardío. Reino Unido impulsó el horror de la Hammer (\emph{Dracula Has Risen from the Grave}, 1968). Estados Unidos consolidó el \emph{exploitation cinema} con obras como \emph{Night of the Living Dead} (Romero, 1968), \emph{The Last House on the Left} (Craven, 1972) y la ola blaxploitation (\emph{Shaft}, Parks, 1971) \citep{Hutchings2004, Koven2006}. 

En este panorama, el cuerpo —fragmentado, estilizado, mutilado o erotizado— se convirtió en un espacio de experimentación estética. El auge del cine pornográfico industrial tras \emph{Deep Throat} (Damiano, 1972) reafirmó una tendencia global hacia la espectacularización del cuerpo, que resonaba de forma distinta en Japón con el \emph{Pinku Eiga} y posteriormente con los \emph{Roman Porno}. Zahlten y Sharp han mostrado cómo estos circuitos de explotación sexual funcionaron como verdaderos laboratorios formales y económicos en paralelo a las producciones de prestigio \citep{Sharp2011, Zahlten2017}.

% ---------------------------------------------------------
\subsection{Crisis del viejo Hollywood y emergencia del New Hollywood}
% ---------------------------------------------------------

La quiebra del sistema clásico estadounidense entre 1967 y 1975 produjo una renovación profunda en los modos de producción y narración. Tras los fracasos millonarios de \emph{Cleopatra} (1963) y \emph{Doctor Dolittle} (1967), los estudios cedieron espacio a jóvenes cineastas formados en escuelas de cine (Coppola, Scorsese, Altman, De Palma), cuyas obras incorporaron violencia estilizada, ambigüedad moral y crítica sociopolítica: \emph{Bonnie and Clyde} (Penn, 1967), \emph{The Graduate} (Nichols, 1967), \emph{Easy Rider} (Hopper, 1969), \emph{Mean Streets} (Scorsese, 1973) o \emph{Carrie} (De Palma, 1976). 

Schatz y Cook han descrito esta fase como un \enquote{interregno} entre el sistema clásico de estudios y el posterior régimen del blockbuster, marcado por presupuestos relativamente moderados, fuerte presencia de personajes marginales y temas contemporáneos, y una mayor autonomía de los directores dentro de los grandes estudios \citep{Schatz1993, Cook2007}. Este clima de libertad expresiva y experimentación formal configura un marco global convergente con los procesos paralelos en Japón, donde la crisis del \textit{studio system} abrió espacios para propuestas híbridas como \emph{Hausu}.

% ---------------------------------------------------------
\subsection{\emph{Jaws} (1975) y el nacimiento del blockbuster global}
% ---------------------------------------------------------

El lanzamiento de \emph{Jaws} (Spielberg, 1975) transformó de forma irreversible el modelo económico del cine comercial internacional. Con un presupuesto estimado en torno a 7--9 millones de dólares y una recaudación mundial que superó los 470 millones, la película se convirtió en el mayor éxito de taquilla de la historia hasta entonces \citep{Wyatt1994, Prince2000}. Más allá de las cifras, su importancia radica en el modelo de distribución y marketing que consolidó: estreno veraniego, campaña televisiva intensiva, lanzamiento simultáneo en cientos de salas y una fuerte explotación de productos derivados.

Wyatt caracteriza \emph{Jaws} como prototipo del \enquote{high concept film}: una premisa fácilmente resumible (\enquote{un tiburón asesino aterroriza una localidad costera}), iconografía poderosa (el póster del tiburón emergiendo bajo la nadadora), y un diseño pensado para generar imágenes reutilizables en tráilers, anuncios y merchandising \citep{Wyatt1994}. Prince subraya que el film combinaba la herencia del cine de autor (uso del fuera de campo, construcción del suspense, tratamiento del trauma comunitario) con una estructura narrativa clásica y un dispositivo de estrellas y efectos especiales que lo hacían accesible al gran público \citep{Prince2000}.

En términos de tendencias industriales, \emph{Jaws} marca el inicio de un desplazamiento del modelo de \emph{roadshow} (estrenos limitados que se expanden lentamente) hacia la llamada \enquote{saturation booking}: un gran número de copias en estreno simultáneo nacional, apoyado en una campaña publicitaria masiva, con el objetivo de maximizar la recaudación en las primeras semanas antes de que opere el boca a boca negativo. Este modelo será perfeccionado dos años después con \emph{Star Wars} (Lucas, 1977) y se convertirá en la norma del blockbuster global en los años ochenta \citep{Schatz1993, Klinger2006}.

El impacto de \emph{Jaws} no se limitó a Estados Unidos. Los análisis de Klinger y de Thompson y Bordwell muestran cómo el éxito del film reconfiguró las expectativas de taquilla en mercados europeos y asiáticos: los distribuidores exigían cada vez más productos capaces de sostener campañas sincronizadas y de generar fenómenos de audiencia concentrados \citep{Klinger2006, BordwellThompson2003}. En Japón, Toho y otros estudios observaron con atención el rendimiento de estos nuevos \emph{event films}. La importación de \emph{Jaws} y, poco después, de \emph{Star Wars} demostró que los blockbusters estadounidenses podían desplazar del primer plano a las producciones nacionales en términos de visibilidad y recaudación, intensificando la sensación de crisis estructural ya presente en el sistema japonés.

En este contexto, \textit{Hausu} puede entenderse como una respuesta local a la lógica del blockbuster global, pero desde una posición lateral. Toho no podía competir con los presupuestos ni con la infraestructura promocional de Spielberg o Lucas, pero sí podía intentar capturar la atención del mismo público juvenil mediante una combinación de terror, humor y espectáculo visual intenso. \textit{Hausu} comparte con \emph{Jaws} ciertos rasgos del cine de evento —un concepto llamativo, una fuerte explotación visual de la casa como dispositivo de muerte, un uso agresivo de los efectos especiales—, pero los desplaza hacia un territorio abiertamente experimental, más cercano a la cultura del anuncio televisivo, el \textit{shōjo manga} y el cine de vanguardia que al clasicismo renovado del New Hollywood.

Así, la hibridez estética de \textit{Hausu} no solo responde a la crisis interna del cine japonés, sino también a la necesidad de negociar con un mercado global en el que el modelo del blockbuster estadounidense se imponía como forma dominante de producción, distribución y consumo cinematográfico.

	\section{Nobuhiko Obayashi: Biografía, Obra, Vanguardias e Influencias}
\label{ch:obayashi}


Nobuhiko Obayashi (1938--2020) ocupa una posición singular en la historia del cine japonés y, por extensión, en el mapa global del cine moderno. Director, montajista, guionista y pionero del cine experimental en 8\,mm y 16\,mm, su trayectoria desborda las categorías habituales del \emph{cine de autor}: transita del cine amateur a los cortometrajes de vanguardia, de ahí a miles de anuncios televisivos, y finalmente a una filmografía de largometrajes que abarca desde el cine juvenil y fantástico hasta un ambicioso tríptico antibélico tardío.\citep{ObayashiSenses2021,CriterionObayashi2020}

Aunque el imaginario popular asocia su nombre casi exclusivamente a \emph{Hausu} (\emph{House}, 1977), la obra de Obayashi forma un corpus coherente en el que se entrecruzan de manera constante memoria, infancia, guerra, cuerpo y tecnología de la imagen. Este capítulo propone una lectura de conjunto de su figura: en primer lugar, se reconstruye su trayectoria biográfica y profesional; en segundo lugar, se esboza una periodización de su filmografía en ciclos temáticos y estilísticos; en tercer lugar, se analizan algunos motivos centrales de su poética visual; finalmente, se revisa la recepción crítica y el lugar que su obra ocupa en las vanguardias audiovisuales japonesas e internacionales.

% =========================================================
\subsection{Trayectoria biográfica y profesional}
% =========================================================

% ---------------------------------------------------------
\subsubsection{Infancia en Onomichi y cine amateur (1938--1960)}
% ---------------------------------------------------------

Obayashi nació el 9 de enero de 1938 en Onomichi, una pequeña ciudad portuaria de la prefectura de Hiroshima. Hijo de médico, creció en un entorno atravesado por la experiencia de la guerra y la posguerra, así como por la presencia temprana de la cultura visual occidental tras la ocupación aliada.\citep{ObayashiSenses2021} La ciudad natal ---sus cuestas, templos y vistas marítimas--- se convertirá décadas más tarde en uno de los espacios topológicos fundamentales de su cine.

Su relación con las imágenes en movimiento comienza muy pronto: a los ocho años recibe de su padre una cámara de 8\,mm con la que realiza dibujos animados caseros y pequeños \emph{home movies}. Durante su etapa universitaria en Tokio (Seijo University), abandona la vía académica convencional y se integra en círculos de cineclub y cine experimental. A mediados de los años cincuenta y primeros sesenta rueda varios cortometrajes en 8\,mm y 16\,mm, entre ellos:

\begin{itemize}
	\item \emph{Complexe} (1964), collage fílmico de 16\,mm que explota la repetición, la superposición y la manipulación manual del celuloide.\citep{Complexe1964}
	\item \emph{Émotion} (1966), subtitulada ``\emph{A Love Story in the Afternoon}''; una pieza de 39 minutos en blanco y negro y color que mezcla melodrama juvenil, vampirismo y cita irónica del cine europeo de arte.\citep{Emotion1966}
\end{itemize}

Junto con otros cineastas experimentales como Takahiko Iimura o Yoichi Takabayashi, Obayashi participa en el colectivo ``Film Independent'' (\emph{Japan Film Andepandan}), cuyos trabajos son mostrados y premiados en festivales de cine experimental europeos a mediados de los sesenta.\citep{ObayashiSenses2021}

% ---------------------------------------------------------
\subsubsection{Del cine experimental a la publicidad televisiva (1960--1977)}
% ---------------------------------------------------------

A partir de la segunda mitad de los años sesenta, Obayashi desplaza progresivamente su actividad hacia el ámbito de la publicidad televisiva. Se convierte en uno de los realizadores de anuncios más prolíficos e innovadores de Japón: se le atribuyen cientos, e incluso miles, de \emph{spots} para marcas como Calpis, Mandom o Hitachi.\citep{MesMidnightEye2009,CriterionObayashi2020}

En este contexto desarrolla varios rasgos que más tarde serán reconocibles en sus largometrajes:

\begin{itemize}
	\item el uso intensivo de \emph{trick photography}, \emph{matte painting} y composiciones de múltiple exposición;
	\item la integración directa de animación, tipografías y gráficos sobre la imagen real;
	\item el empleo de colores extremadamente saturados y esquemas cromáticos pop;
	\item una concepción del montaje como asociación libre, rítmica, cercana al videoclip y al collage publicitario;
	\item un tratamiento lúdico del cuerpo humano, sometido a deformaciones, desapariciones, fragmentaciones o cambios de escala.
\end{itemize}

La publicidad, lejos de representar una mera actividad alimenticia, funciona como laboratorio de formas: allí Obayashi ensaya una gramática audiovisual de alta densidad que, tras el éxito de \emph{Hausu}, transpondrá al largometraje de ficción.

% ---------------------------------------------------------
\subsubsection{Primeros largometrajes y consolidación industrial}
% ---------------------------------------------------------

El debut como director de largometrajes se produce con \emph{Hausu} (Toho, 1977), un encargo en principio concebido como respuesta japonesa al éxito global de \emph{Jaws} (Spielberg, 1975), pero que termina convirtiéndose en un objeto fílmico radicalmente excéntrico dentro del catálogo de la compañía.\citep{House1977,HeartOfWeirdness2018} Ese mismo año dirige también \emph{Hitomi no naka no houmonsha} (\emph{The Visitor in the Eye}, 1977), consolidando su relación profesional con Toho.\citep{ObayashiSenses2021}

A partir de entonces, su actividad en el campo del largometraje se estabiliza y se diversifica: alterna cine juvenil protagonizado por \emph{idols}, relatos fantásticos, dramas literarios y proyectos de fuerte carga autobiográfica o histórica.

% =========================================================
\subsection{Obra cinematográfica: ciclos y etapas}
% =========================================================

La filmografía de Obayashi puede organizarse, con fines analíticos, en varios ciclos o etapas que no son estrictamente cronológicos, pero sí temático-estilísticos.

% ---------------------------------------------------------
\subsubsection{Cortometrajes experimentales (años 50 y 60)}
% ---------------------------------------------------------

Sus trabajos de 8\,mm y 16\,mm de los años cincuenta y sesenta constituyen uno de los núcleos fundacionales del cine experimental japonés de posguerra. Piezas como \emph{Complexe} (1964), \emph{Émotion} (1966) o \emph{Confession} (1968) exploran:

\begin{itemize}
	\item la autorreflexividad del dispositivo cinematográfico;
	\item la hibridación entre melodrama, humor absurdo y cita paródica del cine de terror occidental;
	\item el uso del montaje como ruptura del espacio-tiempo clásico, con saltos bruscos, repeticiones, velocidades alteradas y encadenados imposibles.
\end{itemize}

Estos cortos establecen muchas de las constantes formales de su obra posterior: predominio del truco sobre la ilusión, alegría consciente de la artificialidad y una relación casi táctil con el soporte fílmico.\citep{MesMidnightEye2009}

% ---------------------------------------------------------
\subsubsection{\emph{Hausu} (1977): síntesis de publicidad, horror y cultura pop}
% ---------------------------------------------------------

\emph{Hausu} constituye un punto de inflexión tanto en la carrera de Obayashi como en la historia del cine de horror japonés. El proyecto nace del deseo de Toho de producir un ``\emph{domestic Jaws}'', pero Obayashi plantea la película como traducción audiovisual de los miedos y fantasías de su hija Chigumi, de once años.\citep{HeartOfWeirdness2018,House1977} El resultado es una comedia de horror juvenil en la que siete colegialas visitan la mansión embrujada de una tía soltera y son devoradas ---literal y metafóricamente--- por la casa.

\emph{Hausu} opera simultáneamente en varios niveles:

\begin{enumerate}
	\item \textbf{Relectura del gótico y el \emph{kaidan}}: la casa embrujada y la tía espectral remiten a tradiciones del horror japonés, pero filtradas por el prisma pop y adolescente de los años setenta.
	\item \textbf{Explosión visual anti-realista}: efectos ópticos visibles, cromas imperfectos, animaciones dibujadas, sobreimpresiones y cambios abruptos de escala producen una suerte de \emph{le cinéma du WTF} en clave japonesa.\citep{CriterionObayashi2020}
	\item \textbf{Subtexto histórico y traumático}: la figura de la tía que espera eternamente a su prometido muerto en la guerra permite leer la película como alegoría de un Japón marcado por las ausencias del conflicto y por la domesticación televisiva del trauma.\citep{ObayashiSenses2021}
\end{enumerate}

La película fracasa en su estreno doméstico como producto estrictamente comercial, pero adquiere progresivamente estatus de \emph{cult film}, primero en circuitos cinéfilos japoneses y, desde la restauración y edición de Criterion, en la cinefilia internacional.

% ---------------------------------------------------------
\subsubsection{Cine juvenil, fantasía y el ``ciclo de Onomichi'' (años 80)}
% ---------------------------------------------------------

En los años ochenta, Obayashi deviene uno de los principales especialistas en cine juvenil y \emph{coming-of-age} dentro del sistema industrial japonés.\citep{ObayashiSenses2021,BeyondHouse2020} Entre sus títulos más relevantes destacan:

\begin{itemize}
	\item \emph{Tenkōsei} (\emph{I Are You, You Am Me}, 1982), donde dos adolescentes intercambian literalmente sus cuerpos;
	\item \emph{The Girl Who Leapt Through Time} (1983), adaptación de la novela de Yasutaka Tsutsui, que articula viaje temporal, melodrama y ciencia ficción ligera;
	\item \emph{Bound for the Fields, the Mountains, and the Seacoast} (1986), situada en la preguerra, que vincula infancia, violencia y militarización.
\end{itemize}

Estas obras, muchas de ellas rodadas en Onomichi o en espacios cercanos, suelen agruparse bajo la etiqueta de ``ciclo de Onomichi''. Suelen articular:

\begin{itemize}
	\item una reflexión sobre la memoria local frente a la homogeneización mediática;
	\item un interés por la experiencia adolescente como umbral entre inocencia y violencia histórica;
	\item una poética del espacio cotidiano transformado por la imaginación y el recuerdo.
\end{itemize}

% ---------------------------------------------------------
\subsubsection{Madurez, adaptaciones literarias y exploraciones extremas (1990--2000)}
% ---------------------------------------------------------

En los años noventa, la filmografía de Obayashi se diversifica aún más. Realiza adaptaciones literarias y biográficas que abordan la violencia, la sexualidad y la historia reciente con un tono más sombrío, aunque manteniendo su inclinación por la estilización formal. \emph{Sada} (1998), basada en el célebre caso de Sada Abe, es quizá el ejemplo más conocido: un filme que retoma un episodio emblemático del imaginario erótico-morboso japonés desde una perspectiva que combina distanciamiento formal, subjetividad exacerbada y comentario histórico.\citep{ObayashiSenses2021}

% ---------------------------------------------------------
\subsubsection{La trilogía antibélica contemporánea (2012--2017)}
% ---------------------------------------------------------

En la última etapa de su carrera, marcada por el diagnóstico de un cáncer avanzado, Obayashi consagra tres largometrajes monumentales a una reflexión sobre la guerra, la catástrofe y la responsabilidad intergeneracional:

\begin{itemize}
	\item \emph{Casting Blossoms to the Sky} (2012),
	\item \emph{Seven Weeks} (2014),
	\item \emph{Hanagatami} (2017).\citep{Hanagatami2017,WarTrilogyReview2021}
\end{itemize}

Con duraciones que rondan o superan las tres horas, estos filmes articulan múltiples líneas temporales, capas de memoria y una imaginería intensamente artificial (fondos digitales, composiciones imposibles, superposiciones reiteradas). A través de ellas, Obayashi busca, según sus propias palabras, enviar un mensaje ``de los adultos que vivieron el pasado a los niños del futuro''.\citep{LarkWarTrilogy}

Críticos posteriores han calificado esta trilogía como uno de los logros cinematográficos más singulares de la década de 2010.\citep{LATimesWarTrilogy2021}

% =========================================================
\subsection{Temas, motivos y procedimientos formales}
% =========================================================

% ---------------------------------------------------------
\subsubsection{Tiempo, memoria y nostalgia}
% ---------------------------------------------------------

El tiempo constituye quizá el motivo estructurante por excelencia en la obra de Obayashi. Sus películas vuelven una y otra vez sobre:

\begin{itemize}
	\item la infancia como espacio de memoria reconstruida;
	\item la adolescencia como umbral donde pasado y futuro se superponen;
	\item la guerra como herida temporal que desgarra la continuidad de la vida cotidiana.
\end{itemize}

No se trata de una memoria historicista, sino de una memoria afectiva, mediada por imágenes, relatos familiares, objetos y paisajes. El montaje fragmentario, la superposición de épocas y la inserción de fotografías o material de archivo dentro de la ficción articulan un régimen de tiempo que podríamos llamar \emph{estratigráfico}: varias capas de pasado, presente y futuro coexisten simultáneamente en la pantalla.\citep{ObayashiSenses2021}

% ---------------------------------------------------------
\subsubsection{El cuerpo y el espacio como superficies de inscripción}
% ---------------------------------------------------------

Otro rasgo central es el tratamiento del cuerpo y del espacio como superficies sobre las que se inscriben fuerzas históricas, afectivas y mediáticas. En \emph{Hausu}, el cuerpo de las colegialas se descompone en miembros, cabezas, torsos que flotan, desaparecen o son devorados por objetos domésticos; en las películas de Onomichi, el espacio urbano y marítimo aparece atravesado por capas de recuerdo e imaginación infantil; en la trilogía de guerra, los cuerpos envejecidos y enfermos conviven con la presencia espectral de la juventud perdida.\citep{MesMidnightEye2009,ObayashiSenses2021}

Obayashi rehúye sistemáticamente el realismo anatómico o espacial: sus cuerpos y sus paisajes son plásticos, manipulables, abiertos a la intervención del truco, de la animación y del montaje digital. En ello se aproxima tanto a tradiciones del cine de vanguardia como al imaginario del manga y del anime.

% ---------------------------------------------------------
\subsubsection{Montaje asociativo y artificialidad visible}
% ---------------------------------------------------------

Formalmente, su cine se caracteriza por una apuesta radical por la artificialidad visible:

\begin{itemize}
	\item encadenados y superposiciones que dejan ver las costuras del truco;
	\item fondos pintados, \emph{chroma keys} imperfectos, composiciones de estudio abiertamente teatrales;
	\item montaje que privilegia la asociación poética o emocional por encima de la continuidad espacial clásica.
\end{itemize}

Esta poética del truco asumido establece una posición estética y ética: el cine no es un medio para ``ocultar'' la mediación, sino un espacio para exponerla y jugar con ella, invitando al espectador a participar activamente en la construcción del sentido.

% =========================================================
\subsection{Obayashi y las vanguardias}
% =========================================================

La obra de Obayashi dialoga con varias corrientes vanguardistas, pero mantiene una autonomía marcada.

\begin{itemize}
	\item En relación con la \textbf{Art Theatre Guild} (ATG) y la \emph{Japanese New Wave}, comparte el interés por el anti-realismo, la crítica social y la experimentación formal, pero se distancia por su fuerte impronta pop, su humor lúdico y su preferencia por protagonistas adolescentes.
	\item Frente al \textbf{cine de vanguardia europeo} (Godard, la escuela estructural), su cine es menos ensayístico y más afectivo, anclado en historias concretas y en espacios locales.
	\item En comparación con el \textbf{New Hollywood} y otros cines juveniles internacionales de los setenta, Obayashi adopta la figura de la juventud como sujeto central, pero la sitúa en relación directa con la memoria de la guerra y con la especificidad de la historia japonesa.
\end{itemize}

Esta posición liminar ---entre industria y vanguardia, entre Japón y el imaginario global, entre publicidad y cine de autor--- explica tanto la dificultad inicial para canonizar su obra como la intensidad de su revalorización contemporánea.\citep{BeyondHouse2020}

% =========================================================
\subsection{Recepción crítica, canonización y legado}
% =========================================================

Durante buena parte de su carrera, Obayashi fue percibido por la crítica japonesa como un director excéntrico, asociado al cine juvenil, a la publicidad y a productos genéricamente difíciles de clasificar. La plena canonización de su obra es relativamente tardía y está vinculada, en gran medida, a dos procesos:

\begin{enumerate}
	\item la circulación internacional de \emph{Hausu} en festivales, ciclos de culto y, finalmente, en la edición restaurada de la Criterion Collection;
	\item la recepción crítica entusiasta de su trilogía antibélica en la década de 2010, que permitió releer retrospectivamente toda su filmografía a la luz de una preocupación constante por la memoria y la guerra.\citep{BeyondHouse2020,WarTrilogyReview2021,LWLiesWarTrilogy}
\end{enumerate}

Autores como Tom Mes, Jasper Sharp, Chris Fujiwara o Hal Young han contribuido a situarlo como figura clave del cine japonés moderno, subrayando tanto la coherencia de sus motivos como la singularidad de su lenguaje visual.\citep{MesMidnightEye2009,ObayashiSenses2021} Al mismo tiempo, diversos cineastas contemporáneos ---entre ellos Shunji Iwai, Sion Sono o Takashi Miike--- han reconocido la influencia de su libertad formal y su mezcla de ternura, violencia y humor.\citep{MidnightEyeGuide2004}

En la actualidad, la obra de Nobuhiko Obayashi puede leerse como un verdadero archivo sensible de la posguerra japonesa: un cine que, desde la infancia y la juventud, interroga de manera obstinada el legado de la guerra, la fascinación por la tecnología de la imagen y la capacidad del cine para reinventar, una y otra vez, las formas de recordar y de imaginar el futuro.


	% =========================================================
\section{Fuerzas culturales y mediáticas en Japón (1965--1977)}
\label{ch:fuerzas-culturales}
% =========================================================

El periodo comprendido entre 1965 y 1977 estuvo marcado por una profunda reconfiguración cultural en Japón. Se trató de un momento de aceleración económica, expansión tecnológica, transformaciones en la vida urbana, redefinición de la juventud y proliferación de nuevos medios visuales. Este entramado de factores afectó directamente a la producción cinematográfica y contribuyó a la emergencia de estéticas híbridas, lúdicas y anti-realistas como las que caracterizan la obra de Nobuhiko Obayashi.

La consolidación del llamado ``Milagro Japonés'', articulado en torno al \emph{Income Doubling Plan} de Ikeda, no sólo duplicó el tamaño de la economía en menos de una década, sino que reconfiguró los patrones de consumo y ocio, generando una clase media urbana con creciente acceso a bienes culturales, aparatos electrónicos y productos mediáticos.\citep{Yamamura1987,IncomeDoublingPlan} Paralelamente, la memoria de la guerra y de la ocupación aliada seguía atravesando de forma subterránea la cultura de posguerra, generando tensiones entre pacifismo, amnesia selectiva y revisionismos.\citep{Igarashi2000,Seaton2007}

En este capítulo se examinan seis fuerzas culturales centrales que confluyen en la gestación de \textit{Hausu}: (1) la juventud como nuevo sujeto social, (2) la irrupción masiva de la televisión, (3) la publicidad como laboratorio formal, (4) la cultura pop y el manga como imaginarios transversales, (5) las vanguardias artísticas japonesas y (6) la memoria bélica y el trauma generacional como sustrato ideológico.

% =========================================================
\subsection{Juventud, modernización acelerada y cultura urbana}
% =========================================================

A partir de mediados de los sesenta, Japón experimentó una modernización acelerada impulsada por tasas de crecimiento superiores al 10\% anual y por políticas explícitas de expansión del consumo privado.\citep{IncomeDoublingPlan} El aumento del PIB per cápita vino acompañado de una rápida urbanización: la población residente en áreas metropolitanas como Tokio, Osaka o Nagoya creció de forma sostenida, mientras que el porcentaje de población empleada en sectores agrícolas descendía año tras año. Este proceso generó un nuevo sujeto social: la juventud urbana asalariada o estudiantil, con ingresos disponibles y acceso intensivo a medios de comunicación, moda, música y ocio nocturno.

Desde finales de los cincuenta, las protestas contra el Tratado de Seguridad EE.\,UU.–Japón (ANPO) y las revueltas estudiantiles de 1968--1969 consolidaron a la juventud como actor político central.\citep{Kapur2018,UniversityProtests1968} La ocupación de campus universitarios, las marchas en Shinjuku y los enfrentamientos con la policía no sólo cuestionaron el orden político, sino que instauraron un imaginario visual de la juventud como fuerza de ruptura: cascos, pancartas, performance callejera y cuerpos en movimiento. Este clima de contestación convivió, sin embargo, con una rápida mercantilización de la juventud: revistas, \emph{idols} musicales, motociclistas, colegialas y estudiantes de secundaria se convirtieron en protagonistas de campañas publicitarias, \emph{dramas} televisivos y ciclos cinematográficos.

En el plano de las prácticas culturales, la juventud se convirtió en un mercado segmentado. \citeauthor{Shamoon2012} ha mostrado cómo las revistas y el \textit{shōjo manga} construyeron discursos específicos de ``cultura de chicas'', articulando fantasías de amistad, romance y autonomía juvenil.\citep{Shamoon2012,ShamoonShojoRevolution} Por su parte, \citeauthor{Allison2006} analiza cómo, ya en los setenta, juguetes, \emph{character goods} y productos mediáticos pensados para jóvenes funcionaban como vectores de una imaginación globalizada.\citep{Allison2006} La juventud emerge, así, como sujeto tanto político como consumista, atrapado entre rebeldía y domesticación mercantil.

En \textit{Hausu}, este giro se expresa en la elección de siete colegialas como ejes narrativos, cada una construida como estereotipo pop (\emph{Fantasy}, \emph{Kung-Fu}, \emph{Sweet}, etc.) que encarna valores de juventud, consumo e identidad mediada por imágenes. El hecho de que las protagonistas procedan de un entorno urbano y que la casa embrujada se situe en un espacio rural remite además a la fractura campo/ciudad propia del Japón de alta modernidad: la juventud urbana invade un espacio cargado de memoria bélica y de temporalidades lentas, desencadenando el conflicto generacional.

Una forma de visualizar este proceso de modernización y juvenilización del consumo es comparar la evolución del PIB per cápita y la tasa de urbanización:

\begin{figure}[t]
	\centering
	\begin{tikzpicture}
		\begin{axis}[
			width=\linewidth,
			height=6cm,
			xmin=1960, xmax=1977,
			ymin=0, ymax=350,
			axis y line*=left,
			xlabel={Año},
			ylabel={PIB per cápita (índice 1960=100)},
			xtick={1960,1965,1970,1975},
			ytick={100,150,200,250,300},
			grid=both,
			legend style={at={(0.02,0.98)},anchor=north west,font=\scriptsize},
			]
			\addplot+[mark=*] coordinates {
				(1960,100)
				(1965,170)
				(1970,240)
				(1975,310)
			};
			\addlegendentry{PIB per cápita (real)}
		\end{axis}
		\begin{axis}[
			width=\linewidth,
			height=6cm,
			xmin=1960, xmax=1977,
			ymin=40, ymax=80,
			axis y line*=right,
			axis x line=none,
			ylabel={Población urbana (\%)},
			ytick={40,50,60,70,80},
			]
			\addplot+[mark=square*] coordinates {
				(1960,47)
				(1965,56)
				(1970,63)
				(1975,71)
			};
		\end{axis}
	\end{tikzpicture}
	\caption{Crecimiento económico y urbanización en Japón (1960--1975). Valores indicativos a partir de estimaciones del Banco Mundial.}
	\label{fig:japan_gdp_urban}
\end{figure}

Este gráfico no pretende ofrecer cifras exactas, sino visualizar la simultaneidad entre crecimiento económico, urbanización y emergencia de la juventud como sujeto central de consumo y representación.

% =========================================================
\subsection{Televisión, publicidad y nuevas imágenes}
% =========================================================

Entre 1960 y 1975 la televisión pasó de ser un electrodoméstico de lujo a un aparato presente en prácticamente todos los hogares japoneses. A mediados de los sesenta ya existían unos 30 millones de hogares con televisor, y hacia 1975 la penetración de la televisión en general superaba el 95\%. En el caso de la televisión en color, la transmisión arrancó de forma experimental en 1960 y se consolidó con los Juegos Olímpicos de Tokio de 1964; para 1975, la tasa de adopción de televisores en color había rebasado el 90\%.\citep{TelevisionJapan,TDKTVHistory} Este salto tecnológico reconfiguró el espacio doméstico y el régimen de atención cotidiana.

Los estudios de audiencia de NHK muestran, además, un aumento drástico en el tiempo semanal dedicado a ver televisión: de unas 7 horas semanales en 1960 a más de 20 horas a mediados de los sesenta, con incrementos posteriores que sitúan la televisión como la principal actividad de ocio en el hogar.\citep{Inoue1990,Shiraishi2008} Como señala Inoue, Japón se convirtió en un caso paradigmático de alto consumo televisivo pese a niveles de renta todavía moderados, configurando un país ``hambriento de imágenes'' en medio de un consumo distorsionado.\citep{PatternsBrought}

La televisión alteró directamente la ecología del cine: redujo la asistencia a salas, capturó la publicidad y generó nuevas formas de estrella mediática (presentadores, \emph{idols}, comediantes). Pero también introdujo un nuevo lenguaje visual: montaje rápido, \emph{variety shows}, programas infantiles con decorados kitsch, planos cerrados y uso intensivo de \emph{close-ups}. La frontera entre publicidad, programas infantiles y \emph{tokusatsu} se tornó porosa, con un elevado porcentaje de anuncios animados o con trucajes especiales.\citep{NagaiTVAds, Akiyama1993}

Obayashi se formó precisamente en este entorno. Tras sus experiencias en el cine experimental de 8\,mm y 16\,mm, trabajó durante décadas como realizador de anuncios televisivos, dirigiendo cientos o miles de \emph{CM} para marcas como Mandom, Calpis o Hitachi, a menudo con estrellas internacionales.\citep{MidnightEyeObayashi,CriterionObayashi} Su estilo ---colores saturados, trucajes visibles, montaje fragmentario, humor absurdo--- proviene directamente de este laboratorio publicitario; como ha señalado Roquet, su carrera ilustra cómo los medios mainstream digerían las innovaciones estéticas de la década de 1960.\citep{CriterionObayashi}

\textit{Hausu} reproduce y radicaliza estos códigos televisivos: la artificialidad deliberada de los fondos pintados, las cortinillas gráficas, la inclusión de títulos y rótulos, el uso de música pop y la mezcla de \emph{live action} con animación remiten continuamente a la estética de la televisión comercial. La casa embrujada funciona casi como un \emph{set} televisivo sobredimensionado, donde cada gag o muerte podría leerse como un \emph{sketch} autónomo.

La expansión de la televisión puede visualizarse en términos de penetración y tiempo de visionado:

\begin{figure}[t]
	\centering
	\begin{tikzpicture}
		\begin{axis}[
			width=\linewidth,
			height=6cm,
			xmin=1960, xmax=1975,
			ymin=0, ymax=100,
			xlabel={Año},
			ylabel={Hogares con TV (\%)},
			xtick={1960,1965,1970,1975},
			ytick={0,25,50,75,100},
			grid=both,
			legend style={at={(0.02,0.02)},anchor=south west,font=\scriptsize},
			]
			\addplot+[mark=*] coordinates {
				(1960,55)
				(1965,88)
				(1970,95)
				(1975,98)
			};
			\addlegendentry{Hogares con TV (total)}
		\end{axis}
		\begin{axis}[
			width=\linewidth,
			height=6cm,
			xmin=1960, xmax=1975,
			ymin=0, ymax=30,
			axis y line*=right,
			axis x line=none,
			ylabel={Horas/semana de visionado},
			ytick={0,10,20,30},
			]
			\addplot+[mark=square*] coordinates {
				(1960,7)
				(1965,21)
				(1970,24)
				(1975,26)
			};
		\end{axis}
	\end{tikzpicture}
	\caption{Penetración televisiva y tiempo medio de visionado semanal en Japón (1960--1975). Valores aproximados a partir de encuestas de NHK.}
	\label{fig:japan_tv_usage}
\end{figure}

Este contexto ayuda a entender por qué \textit{Hausu} se percibe, incluso hoy, como una película ``televisiva'' en su textura: está pensada para un público acostumbrado a la intensidad visual y al zapping interno de la programación.

% =========================================================
\subsection{Cultura pop, manga y estética híbrida}
% =========================================================

La cultura pop japonesa de los sesenta y setenta estuvo profundamente influenciada por la expansión del manga comercial, la circulación de imaginarios pop internacionales y el auge de la música juvenil. El manga se consolidó como medio masivo: las tiradas de revistas \emph{shōnen} y \emph{shōjo} crecieron de forma espectacular durante los sesenta y setenta, con \emph{Weekly Shōnen Jump} alcanzando tiradas de varios millones de ejemplares a mediados de los setenta, mientras revistas como \emph{Margaret} o \emph{Nakayoshi} articulaban la cultura de chicas.\citep{Shamoon2012,ShamoonShojoRevolution} El \emph{gekiga} ---manga de tono adulto y realista--- introdujo temas de violencia, erotismo y marginalidad que dialogaban con el cine de explotación.

Paralelamente, el \emph{pop-art} fue reinterpretado por diseñadores como Tadanori Yokoo, cuyas portadas psicodélicas y carteles cinematográficos fusionaban iconografía tradicional japonesa, imágenes publicitarias y estética \emph{sixties} occidental.\citep{YokooDesign} En la música, el movimiento \emph{Group Sounds} y posteriormente el rock japonés crearon escenas juveniles asociadas a Shinjuku y a otros barrios urbanos. Programas televisivos infantiles y de \emph{tokusatsu} ---como \emph{Ultraman} (1966)--- mezclaban acción, fantasía y merchandising, generando un ecosistema transmediático que unía juguetes, series y películas.

Obayashi se sitúa en el epicentro de esta circulación pop. Sus anuncios incorporan recursos gráficos del manga y del diseño pop; su cine tempranamente experimenta con superposiciones, collages y animación. En \textit{Hausu}, esta estética híbrida aparece en múltiples niveles:

\begin{itemize}
	\item los colores hipersaturados y los fondos ilustrados que remiten tanto a decorados teatrales como a páginas de manga coloreadas;
	\item el humor gestual, los \emph{freeze-frames} y las expresiones caricaturescas, próximos a la gramática del \emph{shōnen}/\emph{shōjo};
	\item la fragmentación del cuerpo (cabezas voladoras, miembros seccionados, gatos gigantes) que recuerda al manga de horror y a las ilustraciones de revistas juveniles;
	\item la lógica episódica y acumulativa de la narración, que funciona casi como una sucesión de \emph{chapters} de serial.
\end{itemize}

Mientras que algunos críticos han leído esta estética como simple kitsch o acumulación de recursos sin coherencia,\citep{CriticaKitschHausu} otros han señalado que precisamente esta hibridez permite a \textit{Hausu} funcionar como una especie de ``museo pop'' de la cultura visual japonesa de su tiempo, condensando en 88 minutos un archivo de estilos, modas y fantasías juveniles.\citep{McRoy2008}

Podemos representar de forma esquemática el crecimiento de la cultura impresa juvenil mediante la evolución de algunas revistas clave:

\begin{figure}[t]
	\centering
	\begin{tikzpicture}
		\begin{axis}[
			ybar,
			width=\linewidth,
			height=6cm,
			bar width=10pt,
			ymin=0, ymax=3500,
			xtick=data,
			xticklabel style={rotate=30,anchor=east},
			symbolic x coords={1968,1975},
			ylabel={Tirada aproximada (miles de ejemplares)},
			legend style={at={(0.02,0.98)},anchor=north west,font=\scriptsize},
			grid=both,
			]
			\addplot coordinates {(1968,800) (1975,2500)};
			\addlegendentry{\emph{Shōnen Jump}}
			\addplot coordinates {(1968,400) (1975,800)};
			\addlegendentry{\emph{Margaret}}
			\addplot coordinates {(1968,350) (1975,700)};
			\addlegendentry{\emph{Nakayoshi}}
		\end{axis}
	\end{tikzpicture}
	\caption{Crecimiento aproximado de la tirada de revistas de manga juveniles en Japón (1968--1975). Valores indicativos a partir de estimaciones de la industria.}
	\label{fig:manga_circulation}
\end{figure}

La gráfica subraya el aumento del mercado juvenil impreso, que coexiste con televisión y cine, y que alimenta un imaginario común de personajes, situaciones y emociones del que \textit{Hausu} se nutre de forma explícita.

% =========================================================
\subsection{Movimientos artísticos y vanguardias japonesas}
% =========================================================

La década de 1960 vio el auge de movimientos artísticos radicales que redefinieron el arte japonés y su relación con la política, el cuerpo y el espacio urbano. En el ámbito cinematográfico, la \emph{Art Theatre Guild} (ATG) emergió como plataforma clave para el cine de vanguardia, distribuyendo y produciendo obras de Ōshima, Yoshida, Terayama, Adachi y otros cineastas experimentales.\citep{HarvardATG,MidnightEyeATG} Libres en gran medida de las restricciones de los estudios, estos autores exploraron estructuras narrativas fragmentarias, montaje disyuntivo, erotismo explícito y crítica frontal al Estado y al capitalismo.

Paralelamente, movimientos como el \emph{butō} (Tatsumi Hijikata, Kazuo Ohno) desarrollaron una danza de cuerpos distorsionados, movimientos mínimos y temporalidades dilatadas, en abierta oposición tanto a la danza tradicional como al ballet occidental.\citep{FraleighButoh} Grupos de arte conceptual como \emph{Hi-Red Center} llevaron a cabo intervenciones urbanas y performances que cuestionaban la higiene social, el control y la vida cotidiana en la ciudad de alta modernidad. El teatro de Terayama (\emph{Tenjō Sajiki}) mezcló circo, cabaret, \emph{happenings} y cine en espectáculos multimedia que rompían la cuarta pared y desestabilizaban al espectador.\citep{Gerow2010,SharpUnderground}

Aunque Obayashi no perteneció formalmente a ATG, sí compartió circuitos y sensibilidades con este underground. Sus cortos experimentales en 8\,mm y 16\,mm se proyectaron en espacios alternativos, y su participación en colectivos como el ``Group of Three'' junto a Takabayashi y Iimura evidencia su implicación en la escena.\citep{MakinoATG,ACCObayashi} La diferencia fundamental es que, mientras ATG tendía hacia un cine áspero, abiertamente político y muchas veces pesimista, Obayashi redirige la experimentación formal hacia un registro más lúdico, nostálgico y afectivo.

En \textit{Hausu} pueden rastrearse múltiples huellas de estas vanguardias:

\begin{itemize}
	\item el uso de sobreimpresiones, solarizaciones y trucajes ópticos recuerda al cine experimental de los sesenta;
	\item la fragmentación del cuerpo y las poses extremas de las protagonistas dialogan con ciertas estéticas del \emph{butō};
	\item la ruptura constante del realismo espacial ---habitaciones que cambian de escala, perspectivas imposibles--- conecta con las exploraciones de espacio teatral de Terayama;
	\item la frontalidad de algunas composiciones, con personajes mirando a cámara, hace visible el artificio de la representación.
\end{itemize}

Podemos representar de forma esquemática la centralidad de ATG y de los circuitos alternativos en la década:

\begin{figure}[t]
	\centering
	\begin{tikzpicture}
		\begin{axis}[
			ybar,
			width=\linewidth,
			height=6cm,
			bar width=15pt,
			ymin=0, ymax=25,
			xtick=data,
			symbolic x coords={1961--1965,1966--1970,1971--1975},
			ylabel={N.º aproximado de producciones ATG},
			grid=both,
			]
			\addplot coordinates { (1961--1965,5) (1966--1970,15) (1971--1975,22) };
		\end{axis}
	\end{tikzpicture}
	\caption{Crecimiento aproximado de las producciones de ATG (1961--1975). Valores indicativos basados en recuentos de filmografías de ATG.}
	\label{fig:atg_output}
\end{figure}

Este incremento ilustra cómo, en paralelo al colapso del sistema de estudios, se consolidó una infraestructura alternativa que hizo posible la circulación de formas radicales de cine; \textit{Hausu} puede entenderse como un intento singular de introducir parte de esa radicalidad en el marco de un gran estudio como Toho.

% =========================================================
\subsection{El trauma bélico y la memoria generacional}
% =========================================================

Aunque la cultura pop y juvenil domina la superficie de la época, la memoria de la guerra sigue siendo un elemento central en la identidad japonesa de posguerra. Trabajos como los de \citeauthor{Igarashi2000} y \citeauthor{Seaton2007} han subrayado que las memorias de la guerra en Japón están lejos de ser unívocas: se trata de un campo de disputa entre discursos victimistas, narrativas críticas del militarismo, revisionismos nacionalistas y silencios estructurales.\citep{Igarashi2000,Seaton2007,Penney2024}

En el terreno del cine, la posguerra temprana generó obras de duelo y reconstrucción (\emph{Children of Hiroshima}, Shindō, 1952; \emph{Hiroshima}, Sekigawa, 1953), mientras que los años sesenta y setenta vieron un retorno crítico a la guerra en películas de Okamoto (\emph{Japan's Longest Day}, 1967), Ichikawa (\emph{The Burmese Harp}, 1956; \emph{Fires on the Plain}, 1959) o Imamura (\emph{Black Rain}, 1989, ligeramente posterior pero heredera de estas preocupaciones).\citep{Penney2024,CoatesWarCinema} La televisión, por su parte, contribuyó a institucionalizar memorias victimizantes mediante dramas de prestigio emitidos en fechas conmemorativas.

Obayashi, originario de la prefectura de Hiroshima y niño durante la guerra, incorpora esta dimensión de manera personal y persistente. Como han señalado diversos estudios, su obra está atravesada por una reflexión sobre la infancia, la pérdida y la destrucción nuclear, que se hace explícita en películas posteriores como \emph{The Little Girl Who Conquered Time} (1983) o su trilogía antibélica tardía, pero que ya está presente en la imaginería fantasmática de \textit{Hausu}.\citep{McRoy2008,ObayashiHiroshimaInterview}

En \textit{Hausu}, la tía que espera eternamente a un prometido muerto en la guerra funciona como metáfora de un país atrapado entre trauma y modernización. La casa misma puede leerse como una encarnación espacial del trauma: devora cuerpos jóvenes, absorbe vidas futuras, rehusándose a abandonar el pasado. La oposición entre las protagonistas ---jóvenes, vitales, asociadas a la cultura pop--- y la tía espectral ---atemporal, congelada en el recuerdo de la guerra--- condensa una tensión generacional que ha sido descrita también en otros textos culturales de la época.\citep{Penney2024}

Para situar \textit{Hausu} dentro de esta cartografía de memoria, resulta útil un breve esquema cronológico de algunas obras y acontecimientos clave:

\begin{figure}[t]
	\centering
	\begin{tikzpicture}
		\begin{axis}[
			width=\linewidth,
			height=4.5cm,
			ymin=0, ymax=1,
			xmin=1950, xmax=1980,
			ytick=\empty,
			xtick={1952,1954,1960,1967,1970,1977,1980},
			xlabel={Año},
			grid=both,
			]
			\addplot[only marks,mark=*] coordinates {
				(1952,0.5) % Children of Hiroshima
				(1954,0.5) % Godzilla
				(1960,0.5) % Anpo protests
				(1967,0.5) % Japan's Longest Day
				(1970,0.5) % retorno memoria
				(1977,0.5) % Hausu
			};
			\node[anchor=south] at (axis cs:1952,0.5) {\scriptsize \emph{Children of Hiroshima}};
			\node[anchor=north] at (axis cs:1954,0.5) {\scriptsize \emph{Godzilla}};
			\node[anchor=south] at (axis cs:1960,0.5) {\scriptsize Protestas ANPO};
			\node[anchor=north] at (axis cs:1967,0.5) {\scriptsize \emph{Japan's Longest Day}};
			\node[anchor=south] at (axis cs:1970,0.5) {\scriptsize auge debates memoria};
			\node[anchor=north] at (axis cs:1977,0.5) {\scriptsize \emph{Hausu}};
		\end{axis}
	\end{tikzpicture}
	\caption{Esquema de algunos hitos en la construcción de la memoria bélica en el cine y la cultura japonesa.}
	\label{fig:war_memory_timeline}
\end{figure}

Este esquema sitúa a \textit{Hausu} en un punto intermedio: no es un drama bélico explícito, pero sí un film que desplaza el trauma a un registro fantástico-pop, haciendo que el horror de la guerra se manifieste bajo la forma de una casa voraz que devora a la juventud.

% =========================================================
\subsection{Síntesis: fuerzas culturales que confluyen en \emph{Hausu}}
% =========================================================

\emph{Hausu} se sitúa en el cruce de todas las fuerzas descritas:

\begin{enumerate}
	\item \textbf{Juventud como sujeto visual}: las siete protagonistas encarnan arquetipos pop diseñados para el público adolescente, en sintonía con la mercantilización de la juventud y con la centralidad de estudiantes y colegialas en el imaginario de la época.
	\item \textbf{Televisión y publicidad}: el film adopta la estética rápida, saturada y antinatural de la TV comercial y del \emph{CM}, poniendo en primer plano trucajes visuales, superposiciones y cortinillas gráficas.
	\item \textbf{Cultura pop y manga}: la exageración visual, la lógica episódica y la fragmentación del cuerpo remiten a lenguajes del manga y de programas infantiles de \emph{tokusatsu}, filtrados por la sensibilidad pop-art.
	\item \textbf{Vanguardias japonesas}: la fragmentación de la narración, el collage de estilos y el anti-realismo dialogan con el cine de ATG, el \emph{butō} y el teatro de Terayama, pero traducidos a un registro accesible y lúdico.
	\item \textbf{Memoria bélica}: la tía espectral y la casa como espacio traumático conectan la trama de terror con la herida histórica de la guerra y la hibridez generacional de la posguerra tardía.
\end{enumerate}

Lejos de ser un producto excéntrico y aislado, \emph{Hausu} funciona como una condensación estética de un Japón en plena transformación cultural: entre televisión y memoria, entre modernización y nostalgia, entre juventud y trauma, entre vanguardia y espectáculo. Su singularidad reside precisamente en la capacidad de articular, en un único dispositivo cinematográfico, los lenguajes fragmentarios de la publicidad televisiva, los imaginarios del manga adolescente, las exploraciones formales de la vanguardia y los fantasmas persistentes de la guerra.

Podemos visualizar esta convergencia mediante un diagrama conceptual:

\begin{figure}[t]
	\centering
	\begin{tikzpicture}[node distance=1.8cm]
		\node[draw, rounded corners, align=center, thick] (hausu) {\textbf{\emph{Hausu} (1977)}\\Film híbrido juvenil-pop};
		\node[draw, rounded corners, above left=of hausu, align=center] (youth) {Juventud\\modernización\\urbana};
		\node[draw, rounded corners, above right=of hausu, align=center] (tv) {Televisión\\y publicidad};
		\node[draw, rounded corners, below left=of hausu, align=center] (pop) {Cultura pop\\manga / anime};
		\node[draw, rounded corners, below right=of hausu, align=center] (avant) {Vanguardias\\ATG, butō, teatro};
		\node[draw, rounded corners, below=3.2cm of hausu, align=center] (war) {Memoria bélica\\trauma Hiroshima};
		
		\draw[->, thick] (youth) -- (hausu);
		\draw[->, thick] (tv) -- (hausu);
		\draw[->, thick] (pop) -- (hausu);
		\draw[->, thick] (avant) -- (hausu);
		\draw[->, thick] (war) -- (hausu);
	\end{tikzpicture}
	\caption{Esquema conceptual de las fuerzas culturales y mediáticas que confluyen en \emph{Hausu}.}
	\label{fig:hausu_forces_network}
\end{figure}

Este esquema sintetiza la tesis central del capítulo: \textit{Hausu} no es únicamente una rareza de culto, sino un nodo donde se cruzan tendencias industriales, mediáticas y culturales del Japón de 1965--1977.

% =========================================================
\subsection{Conclusión}
% =========================================================

El Japón de 1965--1977 fue un laboratorio cultural donde convergieron modernización acelerada, masificación televisiva, experimentación artística y redefinición de lo juvenil. La rápida expansión económica y urbana generó nuevas formas de subjetividad y consumo; la televisión y la publicidad reconfiguraron el régimen de imágenes y la economía de la atención; el manga y la cultura pop produjeron imaginarios híbridos de género, cuerpo y deseo; las vanguardias cinematográficas y performativas cuestionaron los límites entre arte y política; y la memoria de la guerra se debatió entre silencios, revisiones críticas y fantasmas persistentes.

Nobuhiko Obayashi, situado en medio de esas fuerzas, desarrolló una poética visual única: lúdica y crítica, pop y melancólica, artesanal y tecnológica. \emph{Hausu} cristaliza de forma particularmente intensa esta poética, funcionando como archivo emocional de la posguerra japonesa y como puente entre vanguardia experimental y cultura popular. Comprender las fuerzas culturales y mediáticas que aquí se han analizado resulta esencial para interpretar no sólo \emph{Hausu}, sino el conjunto de la obra de Obayashi y, más ampliamente, las mutaciones del cine japonés tardomoderno.

	% =========================================================
\section{Influencia del Noh}
\label{ch:noh}
% =========================================================
El teatro \textit{noh} ocupa un lugar central en la imaginación estética japonesa y, al mismo tiempo, en muchas de las formulaciones críticas sobre el cine de terror japonés y su especificidad frente al modelo hollywoodense.\footnote{Sobre la persistencia del \textit{noh} como matriz cultural en el audiovisual japonés contemporáneo, véanse \cite{Brazell2006, Leiter2002}.} En el caso de \textit{Hausu}, la relación con el \textit{noh} no se establece a nivel de cita directa o de adaptación textual, sino a través de una serie de resonancias formales y tematizaciones —el tiempo suspendido, la casa embrujada como espacio liminar, la figura del espíritu femenino vengativo— que conectan el film de Obayashi con una genealogía más amplia de lo fantasmático en la cultura japonesa.

En esta sección se exploran tres dimensiones de esa influencia: (1) la configuración del espacio y el tiempo como ámbitos liminares; (2) la estilización del cuerpo, el gesto y el rostro; y (3) la articulación de lo femenino y lo espectral en clave ritual.

% ---------------------------------------------------------
\subsection{Espacio liminar y tiempo suspendido}
% ---------------------------------------------------------

El \textit{noh} organiza su dramaturgia en torno a espacios liminares: puentes, templos, santuarios, cruces de caminos donde el mundo de los vivos y el de los muertos se superponen. La escenografía mínima y el uso del \textit{hashigakari} (el puente por el que entra el actor principal) construyen un espacio que es menos un lugar real que un umbral donde se materializan recuerdos, resentimientos y apariciones.\citep{Brazell2006}

\textit{Hausu} retoma esta lógica liminar al situar la mayor parte de su acción en una casa de campo que funciona simultáneamente como espacio familiar, tumba y trampa sobrenatural. El viaje de las colegialas desde la ciudad hasta la casa de la tía puede leerse como un tránsito a través de un \enquote{puente} simbólico: el tren, el paisaje pintado, las transiciones ópticas y los fondos artificiales subrayan que el film abandona progresivamente el registro del realismo para entrar en un ámbito regido por otras leyes temporales. La casa, como en muchos dramas de \textit{noh}, no es un simple decorado sino un cuerpo que recuerda, devora y reconfigura a quienes la habitan.

La temporalidad del film también se aproxima a la del \textit{noh}. Aunque la narración de \textit{Hausu} parece avanzar linealmente, las rupturas de continuidad, los flashbacks estilizados (la historia del compromiso truncado de la tía) y las repeticiones de ciertos motivos visuales (la luna, el pozo, el vestido de novia) producen la sensación de un tiempo plegado sobre sí mismo, más cercano al \enquote{tiempo del recuerdo} (\textit{tsuioku}) del \textit{noh} que al tiempo causal del melodrama clásico. El clímax, en el que la casa entera se inunda de sangre y las protagonistas son absorbidas por el espacio, condensa esta lógica de tiempo suspendido: no hay resolución ni retorno a la normalidad, sino una especie de eternización del trauma.

% ---------------------------------------------------------
\subsection{Máscara, rostro y estilización del cuerpo}
% ---------------------------------------------------------

Uno de los elementos más distintivos del \textit{noh} es el uso de máscaras que, lejos de bloquear la expresión, funcionan como superficies altamente codificadas sobre las que pequeños cambios de inclinación o de iluminación generan variaciones dramáticas de emoción.\citep{Leiter2002} La actuación corporal —el \textit{kata}, o patrón de movimiento— y el ritmo vocal sustituyen al psicologismo por una expresividad ritualizada y frontal.

En \textit{Hausu}, Obayashi reinterpreta esta lógica de la máscara a través del dispositivo cinematográfico. Los rostros de las colegialas son constantemente intervenidos por la puesta en escena: congelados en primeros planos de sonrisa publicitaria, recortados por marcos dentro del encuadre, superpuestos con efectos ópticos o deformados por la animación. El célebre momento en que la cabeza decapitada de Kung Fu flota y muerde el trasero de una de sus compañeras no solo funciona como gag gore, sino como inversión grotesca de la máscara \textit{noh}: el rostro se separa del cuerpo, conserva una expresión fija y, al mismo tiempo, adquiere movimiento autónomo.

Algo similar ocurre con el cuerpo. Las poses exageradas, los saltos congelados y la coreografía de los ataques de la casa (el piano que devora a Melody, el reloj que atrapa a Gorgeous) convierten el gesto en figura casi coreográfica, más cercana al \textit{kata} teatral que a la actuación naturalista. En lugar de acceder al interior psicológico de las personajes, el espectador recibe una serie de signos corporales saturados que recuerdan tanto al \textit{noh} como a la publicidad televisiva: cuerpos tratados como signos gráficos, no como individuos profundos.

% ---------------------------------------------------------
\subsection{Lo femenino espectral: de la dama del Noh a la tía de \textit{Hausu}}
% ---------------------------------------------------------

Numerosas piezas de \textit{noh} giran en torno a figuras femeninas espectrales: espíritus de esposas abandonadas, amantes traicionadas o damas de corte consumidas por los celos, cuya aparición ritualizada sirve para reescenificar un trauma histórico o afectivo. Obras como \emph{Aoi no Ue}, \emph{Dōjōji} o \emph{Kinuta} articulan un imaginario donde lo femenino se asocia a la persistencia del rencor (\textit{urami}) y a una forma de violencia que es simultáneamente íntima y cósmica.\citep{Brazell2006, Leiter2002}

La tía de \textit{Hausu} puede leerse en continuidad con esta tradición. Su biografía —una mujer que pierde a su prometido en la guerra, permanece fiel a un vínculo imposible y termina devorando a las jóvenes que visitan su casa— condensa varios motivos del \textit{noh}: el amor no resuelto, la espera interminable, la transformación en espíritu vengativo. Obayashi traduce esta figura a un registro \textit{pop}: en lugar de aparecer con máscara y kimono, lo hace a través de trucajes, transparencias y cambios abruptos de iluminación que la convierten en una presencia inestable, a veces cómica y a veces terrorífica.

El propio diseño de la casa como extensión del cuerpo de la tía —las puertas que se cierran solas, los objetos que atacan, la sangre que inunda las estancias— recuerda a la manera en que, en el \textit{noh}, el espacio escénico se vuelve emanación del estado anímico del espíritu. La fusión final entre Gorgeous y la tía, en la que la sobrina adopta su kimono y su peinado y recibe a su amiga en un plano de calma sobrenatural, funciona como una suerte de \enquote{poscesión hereditaria}: el rencor femenino se transmite y se reactualiza, de forma análoga a como las obras de \textit{noh} reponen una y otra vez las mismas figuras espectrales.

% ---------------------------------------------------------
\subsection{Ritual, repetición y publicidad}
% ---------------------------------------------------------

Si el \textit{noh} se ha descrito a menudo como un teatro de la repetición ritual —el mismo repertorio, los mismos patrones de movimiento, variaciones mínimas en cada función—, la publicidad televisiva se basa en otra forma de repetición: la circulación insistente de eslóganes, jingles e imágenes diseñadas para fijarse en la memoria.\citep{Ivy1995} \textit{Hausu} se sitúa en la intersección de ambos regímenes: toma motivos del imaginario ritual (la casa como santuario, el recuerdo de los muertos, la ofrenda de los cuerpos de las jóvenes) y los recombina siguiendo una lógica de clip publicitario, con cortes abruptos, sobreimpresiones y un uso abiertamente lúdico de los efectos especiales.

En este sentido, la influencia del \textit{noh} en \textit{Hausu} no debe entenderse como una transferencia directa de formas tradicionales a un film de terror, sino como la reactivación, en clave \textit{pop} y televisiva, de ciertas estructuras profundas: el espacio liminar, el tiempo suspendido, la figura del espíritu femenino, la repetición ritualizada del trauma. Obayashi convierte esos elementos en materia de juego visual, pero sin disolver su dimensión inquietante. La película funciona, así, como un \enquote{\textit{noh} eléctrico}: un rito espectral que ha pasado por la lógica de la publicidad y la cultura juvenil de los setenta, pero que sigue organizando la experiencia del espectador en torno a la aparición y la persistencia de lo que no puede ser completamente exorcizado.


	% =========================================================
\section{Qué partes de \textit{Hausu} son intercambiables con otras películas de la época (Japón y circuito global)}
\label{ch:intercambiables}
% =========================================================

Una forma productiva de leer \textit{Hausu} es distinguir entre aquello que la vuelve intercambiable —sus módulos genéricos, su estructura de producción, ciertos tópicos visuales— y aquello que permanece irreductible a cualquier otro film japonés o internacional del periodo. En otras palabras, entre la película como combinación reconocible de piezas de la cultura fílmica de los setenta y la película como objeto singular, resultado de un cruce muy específico entre industria en crisis, cultura juvenil, memoria bélica y experimentación mediática.

En esta sección se señalan cuatro conjuntos de elementos que \textit{Hausu} comparte con el cine de su tiempo (y que podrían haberse reconfigurado en otros contextos) y, en contraste, se argumenta qué dimensiones de la obra resultan difícilmente transferibles a otro film de 1965–1977, tanto en Japón como fuera de él.

% ---------------------------------------------------------
\subsection{Estructura narrativa y dispositivos de género: del \textit{haunted house} al \textit{teen horror}}
% ---------------------------------------------------------

Desde el punto de vista narrativo, \textit{Hausu} se organiza según un esquema ampliamente reconocible: un grupo de adolescentes viaja a un espacio aislado (la casa de la tía en el campo), donde fuerzas sobrenaturales van eliminando a los personajes uno por uno. Este patrón combina dos tradiciones consolidadas:

\begin{enumerate}
	\item el relato de \emph{casa encantada} heredero de \emph{The Haunting} (Wise, 1963) o \emph{The Legend of Hell House} (Hough, 1973);
	\item el emergente \emph{teen horror}, donde cuerpos juveniles funcionan como soporte privilegiado de la violencia y la mutación, como en \emph{Carrie} (De Palma, 1976) o, casi en paralelo, \emph{Suspiria} (Argento, 1977).\footnote{Sobre la centralidad de la adolescencia y el cuerpo femenino en el horror de los setenta, véase \cite{Clover1992, Creed1993}.}
\end{enumerate}

Críticos como Chuck Stephens han subrayado la dimensión de \textit{coming-of-age} del film: siete colegialas estereotipadas —Gorgeous, Fantasy, Kung Fu, Prof, Mac, Melody, Sweet— enfrentan un rito de paso tan letal como carnavalesco, que recuerda a una combinación de \emph{Carrie} multiplicada y un \emph{giallo} lisérgico.\footnote{Stephens describe el film como un “modern masterpiece” del cine de medianoche, articulado en torno a siete adolescentes devoradas por un espíritu femenino, y subraya la importancia de la colaboración de la hija de Obayashi en la concepción del guion. \citep{Stephens2010}.}

En términos estrictamente formales, la estructura puede descomponerse en módulos genéricos intercambiables:

\begin{itemize}
	\item prólogo urbano que presenta conflictos familiares y una decisión impulsiva de viaje;
	\item trayecto hacia el espacio fantástico (el tren, la transición de la ciudad al campo);
	\item llegada lúdica, fase de reconocimiento y falsa seguridad;
	\item secuencia de \enquote{set-pieces} donde cada personaje es confrontado con un dispositivo de muerte ligado a su rasgo identitario (el piano de Melody, la comida de Mac, la limpieza de Sweet, etc.);
	\item clímax de fusión casa–cuerpo–fantasma y epílogo ambiguo.
\end{itemize}

Estos bloques habrían podido sostener un film mucho más convencional —un horror sobrio a la manera de Toho o un derivado directo de \emph{Jaws}— sin que su lógica estructural se viera alterada. De hecho, buena parte del \emph{slasher} norteamericano posterior trabaja con una plantilla casi idéntica, y retrospectivamente \textit{Hausu} parece anticipar, con humor y exceso, fórmulas que se codificarían a finales de los setenta y principios de los ochenta.

% ---------------------------------------------------------
\subsection{Economía de explotación, cine de medianoche y marketing sinérgico}
% ---------------------------------------------------------

También en el plano industrial, \textit{Hausu} comparte rasgos intercambiables con otras producciones de explotación y \emph{cult} de la época. Toho encarga a Obayashi una película que funcione como respuesta local al modelo \emph{blockbuster} inaugurado por \emph{Jaws} (Spielberg, 1975): un \enquote{roller coaster} de terror juvenil capaz de competir con los éxitos de Hollywood que empezaban a dominar las taquillas japonesas.\footnote{Tanto Galbraith como diversas entrevistas con Obayashi recogen que Toho le pidió explícitamente un film “como \emph{Jaws}”, y que el proyecto se concibe desde el inicio como respuesta al ascenso del \emph{blockbuster} estadounidense. \citep{Galbraith2008}.}

Esta lógica de explotación se articula a través de dispositivos fácilmente transferibles a otros productos:

\begin{itemize}
	\item \textbf{Segmentación juvenil}: el film se programa en sesión doble junto a un romance \emph{idol}, dirigido a adolescentes y \emph{office ladies}, replicando estrategias transnacionales de empaquetado genérico similares a las del cine de explotación norteamericano o italiano.
	\item \textbf{Sinergia musical}: la banda Godiego lanza el álbum con temas de \textit{Hausu} antes del estreno, de forma análoga a las campañas de bandas sonoras-rock en el Hollywood de la época.\footnote{La producción de la música por Godiego y Asei Kobayashi se coordinó con el lanzamiento del film, siguiendo un modelo de explotación musical comparable al de otros títulos juveniles del periodo. \citep{Galbraith2008}.}
	\item \textbf{Campaña transmedial}: Obayashi impulsa, incluso antes de tener asegurada la dirección, una novela, un manga y una adaptación radiofónica de \textit{Hausu}; esta expansión previa a la película lo sitúa en el mismo ecosistema de mercantilización transmedia en el que operan otras franquicias juveniles japonesas.
\end{itemize}

La recepción posterior del film como \enquote{midnight movie} —recuperado por cineclubs, ediciones en DVD de culto y retrospectivas— también lo alinea con fenómenos globales como \emph{The Rocky Horror Picture Show} (Sharman, 1975) o \emph{Eraserhead} (Lynch, 1977), donde el valor económico se desplaza de la taquilla inicial a una circulación prolongada en circuitos especializados.

% ---------------------------------------------------------
\subsection{Motivos fantasmales intercambiables: del \textit{kaidan} japonés al gótico global}
% ---------------------------------------------------------

El núcleo fantasmático de \textit{Hausu} también se sostiene sobre motivos ampliamente compartidos con otras tradiciones:

\begin{enumerate}
	\item La tía abandonada en tiempos de guerra, que espera eternamente al prometido fallecido, remite al arquetipo de la novia espectral presente tanto en el \textit{kaidan} japonés (variantes de la figura de Oiwa o de las mujeres-espíritu traicionadas) como en el gótico occidental.\footnote{Sobre la persistencia de la novia espectral y la \textit{onryō} femenina en el cine japonés, véase \cite{McRoy2008, Balmain2008}.}
	\item La casa como organismo devorador, dotado de voluntad propia, puede recogerse en una genealogía que va de ciertos relatos de \emph{kaibyō} (\enquote{gatos fantasma} asociados a mansiones malditas) al horror de mansiones poseídas en el cine internacional.
	\item La mezcla de vampirismo, posesión y animismo felino inserta la película en una constelación más amplia de filmes donde los límites entre humano, animal y arquitectura se difuminan, como en \emph{Black Cat Mansion} (Nakagawa, 1958) o en títulos europeos de horror gótico.
\end{enumerate}

Muchos de estos elementos podrían, en principio, trasplantarse a otra producción de la época sin alterar en exceso su legibilidad genérica. Una tía fantasma hambrienta, un linaje de mujeres sacrificadas por la guerra, una casa que castiga el deseo juvenil: todos son motivos modulables que el cine japonés había trabajado antes y seguiría explorando después, desde el \textit{pink horror} de los setenta hasta el \textit{J-horror} de los noventa.

% ---------------------------------------------------------
\subsection{Lo irreductible de \textit{Hausu}: montaje publicitario, imaginación infantil y exceso pop}
% ---------------------------------------------------------

Frente a esta serie de componentes intercambiables, buena parte de lo que vuelve a \textit{Hausu} un objeto singular parece resistirse a la sustitución. Varias capas se entrecruzan aquí:

\paragraph{a) Visualidad pop-psicodélica y bricolaje mediático.}

Autores como Sarah Cleary han descrito el film como una colisión entre \emph{The Evil Dead} y \emph{Yellow Submarine}: un \enquote{libro desplegable} de horror donde fondos pintados, cromas agresivos, animación sobreimpresa y montaje hiperactivo convierten cada escena en una miniatura autónoma.\footnote{Cleary subraya el carácter de \textit{Hausu} como película “hiper-real” y compara su estética con un libro pop-up psicodélico. \citep{Cleary2020}.}

Este repertorio de trucajes no es simplemente un efecto de época: responde al traslado directo del lenguaje de los anuncios televisivos que Obayashi había perfeccionado en más de dos mil comerciales, incluidos los célebres spots de Mandom con Charles Bronson.\footnote{El perfil de Obayashi como director de publicidad, su relación con Dentsu y el impacto de sus anuncios en la cultura visual japonesa han sido documentados en entrevistas y perfiles críticos. \citep{Graham2018}.} La manera en que el film encadena cortinillas, iris, \emph{split screens}, sobreimpresiones y cambios de textura con una lógica casi \emph{MTV} (antes de MTV) resulta difícil de reproducir incluso en otras películas de explotación contemporáneas.

\paragraph{b) Colaboración con la mirada infantil.}

Diversas fuentes coinciden en que buena parte de los dispositivos de terror —el piano que “muerde” dedos, la cabeza acuática que emerge del pozo, el espejo que devora su reflejo— provienen de las pesadillas y asociaciones libres de Chigumi Obayashi, hija del director, consultada deliberadamente como \emph{co-creadora} conceptual.\footnote{Obayashi declaró que buscaba ideas de su hija porque los adultos sólo piensan en lo que comprenden, mientras que los niños imaginan lo inexplicable. \citep{Galbraith2008}.}

Aunque el cine de terror ha explorado a menudo el imaginario infantil, la integración directa de la imaginación de una niña en la concepción de \textit{Hausu} introduce un tipo de lógica onírica que no coincide exactamente ni con el surrealismo programático de las vanguardias ni con el simbolismo freudiano del horror occidental. El resultado es un tipo de pesadilla lúdica, a medio camino entre cuento ilustrado, juego televisivo y mnemotecnia traumática, que difícilmente podría generarse desde un guionismo profesional estándar.

\paragraph{c) Articulación específica entre trauma bélico y cultura pop.}

Aunque otras películas japonesas de la época abordan la memoria de la guerra, \textit{Hausu} la inscribe en una superficie pop saturada de colores brillantes, música pegadiza y humor absurdo. La tía que nunca dejó de esperar al prometido muerto en combate condensa, en un mismo gesto, la figura de la víctima romántica, la \textit{onryō} vengativa y el residuo no elaborado del trauma atómico, todo ello envuelto en un dispositivo visual cercano al anuncio de refrescos o al programa infantil.\footnote{Jay McRoy y otros autores han leído \textit{Hausu} como una reflexión cifrada sobre la memoria de la guerra en clave pop, en continuidad con la posterior trilogía antibélica de Obayashi. \citep{McRoy2008}.}

Esa combinación entre ligereza formal y densidad histórica es lo que distingue a \textit{Hausu} tanto de las aproximaciones más solemnes al trauma (Kurosawa, Imamura) como de los horrores explotativos que instrumentalizan el cuerpo femenino sin subtexto político claro. En este sentido, \textit{Hausu} no es sólo intercambiable con los \emph{midnight movies} globales: también los subvierte al incrustar, bajo la pirotecnia visual, una meditación melancólica sobre la transmisión generacional del dolor.

% ---------------------------------------------------------
\subsection{Conclusión: modularidad genérica y singularidad autoral}
% ---------------------------------------------------------

Si separamos la película en capas, resulta evidente que \textit{Hausu} comparte —y podría intercambiar— muchos de sus componentes con otras obras del periodo: la estructura \textit{haunted house} + grupo juvenil; la orientación industrial hacia el horror de explotación; el uso de iconografías fantasmales tradicionales; incluso ciertas estrategias de marketing sinérgico.

Sin embargo, la forma específica en que Obayashi acopla esos módulos —su montaje heredado de la publicidad, la incorporación de la imaginación infantil, el cruce entre pop psicodélico y memoria bélica— produce un artefacto que desborda sus condiciones de posibilidad industriales. \textit{Hausu} funciona, así, como un caso límite dentro del cine japonés y global de los setenta: una película construida con piezas intercambiables, pero ensamblada de tal modo que resul
::contentReference[oaicite:11]{index=11}

	% =========================================================
\section{Que partes de Hausus son Irreproducibles con otras peliculas de la epoca tanto Japon como mundial}
\label{ch:irreproducibles}
% =========================================================
	% =========================================================
\section{Contra-ejemplos}
\label{ch:contraejemplos}
% =========================================================
	% =========================================================
\section{Que influencio Hausu, legado}
\label{ch:influencias}
% =========================================================
	% =========================================================
\section{Conclusiones}
\label{ch:influencias}
% =============================
	
	% ============================
	% BIBLIOGRAPHY
	% ============================
	\newpage
	
	% TRUCO TEMPORAL: Fuerza que aparezca todo
	\nocite{*} 
	
	\printbibliography

\end{document}