\section{Introducción}

\textit{Hausu} (1977) ocupa una posición paradójica: es simultáneamente una obra profundamente japonesa y, a la vez, alineada con tendencias internacionales del horror juvenil de los años setenta. Su reputación como film “absurdo” o “inclasificable” suele ocultar la coherencia cultural profunda que sostiene su estética. Este ensayo propone que la rareza de \textit{Hausu} no deriva de su excentricidad formal, sino de la convergencia única de sistemas culturales japoneses que no coincidían en ninguna otra cinematografía mundial.

La hipótesis central de este trabajo sostiene que:

\begin{quote}
\textbf{\textit{Hausu} es irrepetible fuera de Japón porque articula simultáneamente cuatro dimensiones culturales: la ontología fantasmal del teatro Nō, la estética televisiva pop de los años 70, la representación simbólica del cuerpo femenino heredada del \textit{Pink Eiga}, y la posmemoria traumática de Hiroshima procesada a través de códigos infantiles.}
\end{quote}

La metodología combina análisis histórico, comparaciones fílmicas, estudios culturales y evidencia estética cuantitativa.
