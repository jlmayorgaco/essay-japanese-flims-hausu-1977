% =========================================================
\section{Contexto cinematográfico global (1965--1977)}
% =========================================================

El periodo comprendido entre 1965 y 1977 representa una fase de transformación radical del ecosistema cinematográfico mundial. La irrupción de nuevas formas de consumo audiovisual, la consolidación de la televisión como medio dominante, la crisis de los estudios clásicos en Estados Unidos y Europa, el ascenso de movimientos modernistas y contraculturales y la proliferación de cines de explotación contribuyeron a reconfigurar el mapa estético e industrial del cine internacional. Durante estos años se erosionan las bases del clasicismo fílmico, se redefinen los géneros y emergen nuevas sensibilidades que privilegian la subjetividad, el cuerpo, la fragmentación y la experimentación formal \citep{NowellSmith1996, Cook2007}. 

% ---------------------------------------------------------
\subsection{Transformaciones industriales y tecnológicas}
% ---------------------------------------------------------

A mediados de los sesenta, la televisión había alcanzado una presencia masiva en los principales mercados occidentales, reduciendo la asistencia a salas, forzando el cierre de cines y obligando a los estudios a reconfigurar sus estrategias de producción. En Estados Unidos, la asistencia anual cayó de 90 millones de espectadores semanales en 1948 a menos de 17 millones en 1967 \citep{Gomery1992}. Europa siguió un patrón similar, con descensos significativos en Francia, Italia y Reino Unido; por ejemplo, la asistencia británica se redujo de 1{,}400 millones de entradas en 1950 a 176 millones en 1965 \citep{Harper2004}. 

La industria respondió con diversas estrategias: formatos panorámicos, color como estándar, cine espectáculo, y un aumento considerable en producciones de bajo presupuesto. Los mercados comenzaron a polarizarse entre grandes producciones de estudio (épicos, musicales, cine catástrofe) y nichos de explotación (horror, erotismo, ciencia ficción menor), fenómeno que se observa tanto en Hollywood como en Italia, España, Francia o Alemania Occidental \citep{Hutchings2004}.

% ---------------------------------------------------------
\subsection{Movimientos modernistas y contraculturales}
% ---------------------------------------------------------

El período 1965--1977 coincide con el auge global del cine modernista, que redefinió el lenguaje fílmico mediante fragmentación narrativa, reflexividad, subjetividad extrema y rupturas espaciales y temporales. La Nouvelle Vague seguía produciendo obras clave —como \emph{Pierrot le fou} (Godard, 1965) o \emph{La Chinoise} (Godard, 1967)— que exploraban la relación entre política, juventud y medios de comunicación. En Italia, Antonioni continuó la investigación de la alienación moderna con \emph{Blow-Up} (1966) y \emph{Zabriskie Point} (1970). En Alemania, el Nuevo Cine Alemán consolidó figuras como Fassbinder (\emph{Liebe ist kälter als der Tod}, 1969), Herzog (\emph{Aguirre}, 1972) y Wenders (\emph{Der amerikanische Freund}, 1977) \citep{Elsaesser1989}.

Estos movimientos compartían un interés por la desestabilización del realismo, la exploración de estados psicogeográficos y la crítica cultural; un horizonte que dialoga, aunque desde coordenadas distintas, con el anti-realismo y la imaginería pop-publicitaria que Obayashi desarrollaría en \textit{Hausu}.

% ---------------------------------------------------------
\subsection{Géneros populares y cines de explotación}
% ---------------------------------------------------------

Simultáneamente, la década vio una proliferación internacional de cines populares y de explotación, frecuentemente producidos con bajo presupuesto y dirigidos a públicos juveniles. Italia desarrolló el \emph{giallo} (Argento, \emph{L'uccello dalle piume di cristallo}, 1970), los \emph{poliziotteschi} (\emph{La polizia incrimina, la legge assolve}, Lenzi, 1973) y el horror gótico tardío. Reino Unido impulsó el horror de la Hammer (\emph{Dracula Has Risen from the Grave}, 1968). Estados Unidos consolidó el \emph{exploitation cinema} con obras como \emph{Night of the Living Dead} (Romero, 1968), \emph{The Last House on the Left} (Craven, 1972) y la ola blaxploitation (\emph{Shaft}, Parks, 1971). 

En este panorama, el cuerpo —fragmentado, estilizado, mutilado o erotizado— se convirtió en un espacio de experimentación estética. El auge del cine pornográfico industrial tras \emph{Deep Throat} (Damiano, 1972) reafirmó una tendencia global hacia la espectacularización del cuerpo, que resonaba de forma distinta en Japón con el \emph{Pink Eiga} y posteriormente con los \emph{Roman Porno}. 

% ---------------------------------------------------------
\subsection{Crisis del viejo Hollywood y emergencia del New Hollywood}
% ---------------------------------------------------------

La quiebra del sistema clásico estadounidense entre 1967 y 1975 produjo una renovación profunda en los modos de producción y narración. Tras los fracasos millonarios de \emph{Cleopatra} (1963) y \emph{Doctor Dolittle} (1967), los estudios cedieron espacio a jóvenes cineastas formados en escuelas de cine (Coppola, Scorsese, Altman, De Palma), cuyas obras incorporaron violencia estilizada, ambigüedad moral y crítica sociopolítica: \emph{Bonnie and Clyde} (Penn, 1967), \emph{The Graduate} (Nichols, 1967), \emph{Mean Streets} (Scorsese, 1973) o \emph{Carrie} (De Palma, 1976). Este clima de libertad expresiva y experimentación formal configura un marco global convergente con los procesos paralelos en Japón, donde la crisis del \textit{studio system} abrió espacios para propuestas híbridas como \emph{Hausu}.

% ---------------------------------------------------------
\subsection{\emph{Jaws} (1975) y el nacimiento del blockbuster global}
% ---------------------------------------------------------

El lanzamiento de \emph{Jaws} (Spielberg, 1975) transformó de forma irreversible el modelo económico del cine comercial internacional. El uso de campañas publicitarias masivas, estrenos simultáneos nacionales y apoyo televisivo configuró el llamado ``modelo blockbuster''. Su éxito inmediato influyó en las decisiones de diversos mercados asiáticos. En Japón, Toho identificó en \emph{Jaws} la oportunidad de producir un ``horror juvenil'' que resonara con el público adolescente contemporáneo. En este contexto industrial se inscribe la producción de \textit{Hausu}, cuya hibridez estética —entre terror, humor, pop-art y fantasía televisiva— aparece como respuesta local a las transformaciones globales del mercado \citep{Wyatt1994, Prince2000}.
