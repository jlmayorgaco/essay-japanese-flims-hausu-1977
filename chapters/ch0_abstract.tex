\section*{Resumen}

\noindent
\textit{Hausu} (Nobuhiko Obayashi, 1977) ha sido tradicionalmente descrita como un film de culto “inclasificable” debido a su estética saturada, su mezcla de horror y comedia y su montaje fragmentado. Este ensayo sostiene que la película no puede comprenderse únicamente desde las tendencias globales del proto-slasher ni desde el experimentalismo japonés de la época. Por el contrario, \textit{Hausu} emerge como el resultado singular de la convergencia de cuatro fuerzas culturales específicas del Japón de los años 1970: (1) la dramaturgia del fantasma femenino del teatro Nō; (2) la gramática visual hipersaturada de la televisión y la publicidad japonesa; (3) la lógica representacional del cuerpo femenino heredada del \textit{Pink Eiga} y el \textit{Roman Porno}; y (4) la memoria traumática pos-Hiroshima procesada mediante códigos pop-infantiles. A través de la comparación con obras europeas y estadounidenses contemporáneas, el ensayo identifica qué elementos son intercambiables y cuáles resultan irrepetibles fuera de Japón. La conclusión afirma que \textit{Hausu} constituye una singularidad estética que ilumina el cruce entre tradición, modernidad pop y trauma histórico en el Japón posbélico.





\begin{figure}[ht]
	\centering
	\begin{tikzpicture}[
		node distance=8mm,
		% Estilo de las cajas (Sin el comando title erróneo)
		box/.style={
			rectangle,
			rounded corners,
			draw=hausuDark,
			thick,
			fill=hausuGhost,
			align=left,
			font=\small\sffamily,
			inner sep=6pt,
			text width=0.75\linewidth,
			drop shadow={opacity=0.5, shadow xshift=2pt, shadow yshift=-2pt}
		},
		% Estilo del núcleo
		core/.style={
			rectangle,
			rounded corners,
			draw=hausuDark,
			ultra thick,
			fill=hausuBlood,
			text=white,
			align=center,
			font=\bfseries\large\sffamily,
			inner sep=8pt,
			text width=0.6\linewidth,
			drop shadow={opacity=0.8, shadow xshift=3pt, shadow yshift=-3pt}
		},
		% Flechas
		arrow/.style={-Latex, color=hausuOrange, line width=2pt}
		]
		
		% ---- Entradas japonesas ----
		% CORRECCIÓN: Quitamos "title=..." de aquí
		\node[box] (jp) {
			\textbf{\textcolor{hausuPurple}{Entradas japonesas}}\\[3pt]
			-- Teatro Nō: fantasma femenino, tiempo suspendido.\\
			-- TV y publicidad 70s: colores saturados, ritmo frenético.\\
			-- \textit{Pink Eiga}: cuerpo femenino fragmentado.\\
			-- Trauma pos-Hiroshima: memoria bélica y fantasía.
		};
		
		% ---- Entradas globales ----
		\node[box, below=of jp] (gl) {
			\textbf{\textcolor{hausuPurple}{Entradas globales}}\\[3pt]
			-- Horror juvenil / proto-slasher.\\
			-- Surrealismo sensorial europeo.\\
			-- Cine experimental 70s: anti-realismo.\\
			-- Efectos ópticos y trucos psicodélicos.
		};
		
		% ---- Nodo central: Hausu ----
		\node[core, below=of gl] (hausu) {
			\faCat \space \textit{HAUSU} (1977) \space \faGhost \\[3pt]
			\small Convergencia de Japón + Global
		};
		
		% ---- Salida / resultado estético ----
		\node[box, below=of hausu] (out) {
			\textbf{\textcolor{hausuPurple}{Salida: estética y legado}}\\[3pt]
			-- Horror infantil-pop y \textit{kawaii} siniestro.\\
			-- Lógica de ``videojuego'' (niveles, roles).\\
			-- Influencia en horror pop y anime posterior.
		};
		
		% Flechas
		\draw[arrow] (jp.south) -- (gl.north);
		\draw[arrow] (gl.south) -- (hausu.north);
		\draw[arrow] (hausu.south) -- (out.north);
		
	\end{tikzpicture}
	\caption{Esquema de la singularidad estética de \textit{Hausu}.}
\end{figure}