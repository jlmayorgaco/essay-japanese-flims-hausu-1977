% =========================================================
\section{Qué partes de \textit{Hausu} son intercambiables con otras películas de la época (Japón y circuito global)}
\label{ch:intercambiables}
% =========================================================

Una forma productiva de leer \textit{Hausu} es distinguir entre aquello que la vuelve intercambiable —sus módulos genéricos, su estructura de producción, ciertos tópicos visuales— y aquello que permanece irreductible a cualquier otro film japonés o internacional del periodo. En otras palabras, entre la película como combinación reconocible de piezas de la cultura fílmica de los setenta y la película como objeto singular, resultado de un cruce muy específico entre industria en crisis, cultura juvenil, memoria bélica y experimentación mediática.

En esta sección se señalan cuatro conjuntos de elementos que \textit{Hausu} comparte con el cine de su tiempo (y que podrían haberse reconfigurado en otros contextos) y, en contraste, se argumenta qué dimensiones de la obra resultan difícilmente transferibles a otro film de 1965–1977, tanto en Japón como fuera de él.

% ---------------------------------------------------------
\subsection{Estructura narrativa y dispositivos de género: del \textit{haunted house} al \textit{teen horror}}
% ---------------------------------------------------------

Desde el punto de vista narrativo, \textit{Hausu} se organiza según un esquema ampliamente reconocible: un grupo de adolescentes viaja a un espacio aislado (la casa de la tía en el campo), donde fuerzas sobrenaturales van eliminando a los personajes uno por uno. Este patrón combina dos tradiciones consolidadas:

\begin{enumerate}
	\item el relato de \emph{casa encantada} heredero de \emph{The Haunting} (Wise, 1963) o \emph{The Legend of Hell House} (Hough, 1973);
	\item el emergente \emph{teen horror}, donde cuerpos juveniles funcionan como soporte privilegiado de la violencia y la mutación, como en \emph{Carrie} (De Palma, 1976) o, casi en paralelo, \emph{Suspiria} (Argento, 1977).\footnote{Sobre la centralidad de la adolescencia y el cuerpo femenino en el horror de los setenta, véase \cite{Clover1992, Creed1993}.}
\end{enumerate}

Críticos como Chuck Stephens han subrayado la dimensión de \textit{coming-of-age} del film: siete colegialas estereotipadas —Gorgeous, Fantasy, Kung Fu, Prof, Mac, Melody, Sweet— enfrentan un rito de paso tan letal como carnavalesco, que recuerda a una combinación de \emph{Carrie} multiplicada y un \emph{giallo} lisérgico.\footnote{Stephens describe el film como un “modern masterpiece” del cine de medianoche, articulado en torno a siete adolescentes devoradas por un espíritu femenino, y subraya la importancia de la colaboración de la hija de Obayashi en la concepción del guion. \citep{Stephens2010}.}

En términos estrictamente formales, la estructura puede descomponerse en módulos genéricos intercambiables:

\begin{itemize}
	\item prólogo urbano que presenta conflictos familiares y una decisión impulsiva de viaje;
	\item trayecto hacia el espacio fantástico (el tren, la transición de la ciudad al campo);
	\item llegada lúdica, fase de reconocimiento y falsa seguridad;
	\item secuencia de \enquote{set-pieces} donde cada personaje es confrontado con un dispositivo de muerte ligado a su rasgo identitario (el piano de Melody, la comida de Mac, la limpieza de Sweet, etc.);
	\item clímax de fusión casa–cuerpo–fantasma y epílogo ambiguo.
\end{itemize}

Estos bloques habrían podido sostener un film mucho más convencional —un horror sobrio a la manera de Toho o un derivado directo de \emph{Jaws}— sin que su lógica estructural se viera alterada. De hecho, buena parte del \emph{slasher} norteamericano posterior trabaja con una plantilla casi idéntica, y retrospectivamente \textit{Hausu} parece anticipar, con humor y exceso, fórmulas que se codificarían a finales de los setenta y principios de los ochenta.

% ---------------------------------------------------------
\subsection{Economía de explotación, cine de medianoche y marketing sinérgico}
% ---------------------------------------------------------

También en el plano industrial, \textit{Hausu} comparte rasgos intercambiables con otras producciones de explotación y \emph{cult} de la época. Toho encarga a Obayashi una película que funcione como respuesta local al modelo \emph{blockbuster} inaugurado por \emph{Jaws} (Spielberg, 1975): un \enquote{roller coaster} de terror juvenil capaz de competir con los éxitos de Hollywood que empezaban a dominar las taquillas japonesas.\footnote{Tanto Galbraith como diversas entrevistas con Obayashi recogen que Toho le pidió explícitamente un film “como \emph{Jaws}”, y que el proyecto se concibe desde el inicio como respuesta al ascenso del \emph{blockbuster} estadounidense. \citep{Galbraith2008}.}

Esta lógica de explotación se articula a través de dispositivos fácilmente transferibles a otros productos:

\begin{itemize}
	\item \textbf{Segmentación juvenil}: el film se programa en sesión doble junto a un romance \emph{idol}, dirigido a adolescentes y \emph{office ladies}, replicando estrategias transnacionales de empaquetado genérico similares a las del cine de explotación norteamericano o italiano.
	\item \textbf{Sinergia musical}: la banda Godiego lanza el álbum con temas de \textit{Hausu} antes del estreno, de forma análoga a las campañas de bandas sonoras-rock en el Hollywood de la época.\footnote{La producción de la música por Godiego y Asei Kobayashi se coordinó con el lanzamiento del film, siguiendo un modelo de explotación musical comparable al de otros títulos juveniles del periodo. \citep{Galbraith2008}.}
	\item \textbf{Campaña transmedial}: Obayashi impulsa, incluso antes de tener asegurada la dirección, una novela, un manga y una adaptación radiofónica de \textit{Hausu}; esta expansión previa a la película lo sitúa en el mismo ecosistema de mercantilización transmedia en el que operan otras franquicias juveniles japonesas.
\end{itemize}

La recepción posterior del film como \enquote{midnight movie} —recuperado por cineclubs, ediciones en DVD de culto y retrospectivas— también lo alinea con fenómenos globales como \emph{The Rocky Horror Picture Show} (Sharman, 1975) o \emph{Eraserhead} (Lynch, 1977), donde el valor económico se desplaza de la taquilla inicial a una circulación prolongada en circuitos especializados.

% ---------------------------------------------------------
\subsection{Motivos fantasmales intercambiables: del \textit{kaidan} japonés al gótico global}
% ---------------------------------------------------------

El núcleo fantasmático de \textit{Hausu} también se sostiene sobre motivos ampliamente compartidos con otras tradiciones:

\begin{enumerate}
	\item La tía abandonada en tiempos de guerra, que espera eternamente al prometido fallecido, remite al arquetipo de la novia espectral presente tanto en el \textit{kaidan} japonés (variantes de la figura de Oiwa o de las mujeres-espíritu traicionadas) como en el gótico occidental.\footnote{Sobre la persistencia de la novia espectral y la \textit{onryō} femenina en el cine japonés, véase \cite{McRoy2008, Balmain2008}.}
	\item La casa como organismo devorador, dotado de voluntad propia, puede recogerse en una genealogía que va de ciertos relatos de \emph{kaibyō} (\enquote{gatos fantasma} asociados a mansiones malditas) al horror de mansiones poseídas en el cine internacional.
	\item La mezcla de vampirismo, posesión y animismo felino inserta la película en una constelación más amplia de filmes donde los límites entre humano, animal y arquitectura se difuminan, como en \emph{Black Cat Mansion} (Nakagawa, 1958) o en títulos europeos de horror gótico.
\end{enumerate}

Muchos de estos elementos podrían, en principio, trasplantarse a otra producción de la época sin alterar en exceso su legibilidad genérica. Una tía fantasma hambrienta, un linaje de mujeres sacrificadas por la guerra, una casa que castiga el deseo juvenil: todos son motivos modulables que el cine japonés había trabajado antes y seguiría explorando después, desde el \textit{pink horror} de los setenta hasta el \textit{J-horror} de los noventa.

% ---------------------------------------------------------
\subsection{Lo irreductible de \textit{Hausu}: montaje publicitario, imaginación infantil y exceso pop}
% ---------------------------------------------------------

Frente a esta serie de componentes intercambiables, buena parte de lo que vuelve a \textit{Hausu} un objeto singular parece resistirse a la sustitución. Varias capas se entrecruzan aquí:

\paragraph{a) Visualidad pop-psicodélica y bricolaje mediático.}

Autores como Sarah Cleary han descrito el film como una colisión entre \emph{The Evil Dead} y \emph{Yellow Submarine}: un \enquote{libro desplegable} de horror donde fondos pintados, cromas agresivos, animación sobreimpresa y montaje hiperactivo convierten cada escena en una miniatura autónoma.\footnote{Cleary subraya el carácter de \textit{Hausu} como película “hiper-real” y compara su estética con un libro pop-up psicodélico. \citep{Cleary2020}.}

Este repertorio de trucajes no es simplemente un efecto de época: responde al traslado directo del lenguaje de los anuncios televisivos que Obayashi había perfeccionado en más de dos mil comerciales, incluidos los célebres spots de Mandom con Charles Bronson.\footnote{El perfil de Obayashi como director de publicidad, su relación con Dentsu y el impacto de sus anuncios en la cultura visual japonesa han sido documentados en entrevistas y perfiles críticos. \citep{Graham2018}.} La manera en que el film encadena cortinillas, iris, \emph{split screens}, sobreimpresiones y cambios de textura con una lógica casi \emph{MTV} (antes de MTV) resulta difícil de reproducir incluso en otras películas de explotación contemporáneas.

\paragraph{b) Colaboración con la mirada infantil.}

Diversas fuentes coinciden en que buena parte de los dispositivos de terror —el piano que “muerde” dedos, la cabeza acuática que emerge del pozo, el espejo que devora su reflejo— provienen de las pesadillas y asociaciones libres de Chigumi Obayashi, hija del director, consultada deliberadamente como \emph{co-creadora} conceptual.\footnote{Obayashi declaró que buscaba ideas de su hija porque los adultos sólo piensan en lo que comprenden, mientras que los niños imaginan lo inexplicable. \citep{Galbraith2008}.}

Aunque el cine de terror ha explorado a menudo el imaginario infantil, la integración directa de la imaginación de una niña en la concepción de \textit{Hausu} introduce un tipo de lógica onírica que no coincide exactamente ni con el surrealismo programático de las vanguardias ni con el simbolismo freudiano del horror occidental. El resultado es un tipo de pesadilla lúdica, a medio camino entre cuento ilustrado, juego televisivo y mnemotecnia traumática, que difícilmente podría generarse desde un guionismo profesional estándar.

\paragraph{c) Articulación específica entre trauma bélico y cultura pop.}

Aunque otras películas japonesas de la época abordan la memoria de la guerra, \textit{Hausu} la inscribe en una superficie pop saturada de colores brillantes, música pegadiza y humor absurdo. La tía que nunca dejó de esperar al prometido muerto en combate condensa, en un mismo gesto, la figura de la víctima romántica, la \textit{onryō} vengativa y el residuo no elaborado del trauma atómico, todo ello envuelto en un dispositivo visual cercano al anuncio de refrescos o al programa infantil.\footnote{Jay McRoy y otros autores han leído \textit{Hausu} como una reflexión cifrada sobre la memoria de la guerra en clave pop, en continuidad con la posterior trilogía antibélica de Obayashi. \citep{McRoy2008}.}

Esa combinación entre ligereza formal y densidad histórica es lo que distingue a \textit{Hausu} tanto de las aproximaciones más solemnes al trauma (Kurosawa, Imamura) como de los horrores explotativos que instrumentalizan el cuerpo femenino sin subtexto político claro. En este sentido, \textit{Hausu} no es sólo intercambiable con los \emph{midnight movies} globales: también los subvierte al incrustar, bajo la pirotecnia visual, una meditación melancólica sobre la transmisión generacional del dolor.

% ---------------------------------------------------------
\subsection{Conclusión: modularidad genérica y singularidad autoral}
% ---------------------------------------------------------

Si separamos la película en capas, resulta evidente que \textit{Hausu} comparte —y podría intercambiar— muchos de sus componentes con otras obras del periodo: la estructura \textit{haunted house} + grupo juvenil; la orientación industrial hacia el horror de explotación; el uso de iconografías fantasmales tradicionales; incluso ciertas estrategias de marketing sinérgico.

Sin embargo, la forma específica en que Obayashi acopla esos módulos —su montaje heredado de la publicidad, la incorporación de la imaginación infantil, el cruce entre pop psicodélico y memoria bélica— produce un artefacto que desborda sus condiciones de posibilidad industriales. \textit{Hausu} funciona, así, como un caso límite dentro del cine japonés y global de los setenta: una película construida con piezas intercambiables, pero ensamblada de tal modo que resul
::contentReference[oaicite:11]{index=11}
