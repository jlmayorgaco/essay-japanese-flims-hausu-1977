\section{Nobuhiko Obayashi: Biografía, Obra, Vanguardias e Influencias}
\label{ch:obayashi}


Nobuhiko Obayashi (1938--2020) ocupa una posición singular en la historia del cine japonés y, por extensión, en el mapa global del cine moderno. Director, montajista, guionista y pionero del cine experimental en 8\,mm y 16\,mm, su trayectoria desborda las categorías habituales del \emph{cine de autor}: transita del cine amateur a los cortometrajes de vanguardia, de ahí a miles de anuncios televisivos, y finalmente a una filmografía de largometrajes que abarca desde el cine juvenil y fantástico hasta un ambicioso tríptico antibélico tardío.\citep{ObayashiSenses2021,CriterionObayashi2020}

Aunque el imaginario popular asocia su nombre casi exclusivamente a \emph{Hausu} (\emph{House}, 1977), la obra de Obayashi forma un corpus coherente en el que se entrecruzan de manera constante memoria, infancia, guerra, cuerpo y tecnología de la imagen. Este capítulo propone una lectura de conjunto de su figura: en primer lugar, se reconstruye su trayectoria biográfica y profesional; en segundo lugar, se esboza una periodización de su filmografía en ciclos temáticos y estilísticos; en tercer lugar, se analizan algunos motivos centrales de su poética visual; finalmente, se revisa la recepción crítica y el lugar que su obra ocupa en las vanguardias audiovisuales japonesas e internacionales.

% =========================================================
\subsection{Trayectoria biográfica y profesional}
% =========================================================

% ---------------------------------------------------------
\subsubsection{Infancia en Onomichi y cine amateur (1938--1960)}
% ---------------------------------------------------------

Obayashi nació el 9 de enero de 1938 en Onomichi, una pequeña ciudad portuaria de la prefectura de Hiroshima. Hijo de médico, creció en un entorno atravesado por la experiencia de la guerra y la posguerra, así como por la presencia temprana de la cultura visual occidental tras la ocupación aliada.\citep{ObayashiSenses2021} La ciudad natal ---sus cuestas, templos y vistas marítimas--- se convertirá décadas más tarde en uno de los espacios topológicos fundamentales de su cine.

Su relación con las imágenes en movimiento comienza muy pronto: a los ocho años recibe de su padre una cámara de 8\,mm con la que realiza dibujos animados caseros y pequeños \emph{home movies}. Durante su etapa universitaria en Tokio (Seijo University), abandona la vía académica convencional y se integra en círculos de cineclub y cine experimental. A mediados de los años cincuenta y primeros sesenta rueda varios cortometrajes en 8\,mm y 16\,mm, entre ellos:

\begin{itemize}
	\item \emph{Complexe} (1964), collage fílmico de 16\,mm que explota la repetición, la superposición y la manipulación manual del celuloide.\citep{Complexe1964}
	\item \emph{Émotion} (1966), subtitulada ``\emph{A Love Story in the Afternoon}''; una pieza de 39 minutos en blanco y negro y color que mezcla melodrama juvenil, vampirismo y cita irónica del cine europeo de arte.\citep{Emotion1966}
\end{itemize}

Junto con otros cineastas experimentales como Takahiko Iimura o Yoichi Takabayashi, Obayashi participa en el colectivo ``Film Independent'' (\emph{Japan Film Andepandan}), cuyos trabajos son mostrados y premiados en festivales de cine experimental europeos a mediados de los sesenta.\citep{ObayashiSenses2021}

% ---------------------------------------------------------
\subsubsection{Del cine experimental a la publicidad televisiva (1960--1977)}
% ---------------------------------------------------------

A partir de la segunda mitad de los años sesenta, Obayashi desplaza progresivamente su actividad hacia el ámbito de la publicidad televisiva. Se convierte en uno de los realizadores de anuncios más prolíficos e innovadores de Japón: se le atribuyen cientos, e incluso miles, de \emph{spots} para marcas como Calpis, Mandom o Hitachi.\citep{MesMidnightEye2009,CriterionObayashi2020}

En este contexto desarrolla varios rasgos que más tarde serán reconocibles en sus largometrajes:

\begin{itemize}
	\item el uso intensivo de \emph{trick photography}, \emph{matte painting} y composiciones de múltiple exposición;
	\item la integración directa de animación, tipografías y gráficos sobre la imagen real;
	\item el empleo de colores extremadamente saturados y esquemas cromáticos pop;
	\item una concepción del montaje como asociación libre, rítmica, cercana al videoclip y al collage publicitario;
	\item un tratamiento lúdico del cuerpo humano, sometido a deformaciones, desapariciones, fragmentaciones o cambios de escala.
\end{itemize}

La publicidad, lejos de representar una mera actividad alimenticia, funciona como laboratorio de formas: allí Obayashi ensaya una gramática audiovisual de alta densidad que, tras el éxito de \emph{Hausu}, transpondrá al largometraje de ficción.

% ---------------------------------------------------------
\subsubsection{Primeros largometrajes y consolidación industrial}
% ---------------------------------------------------------

El debut como director de largometrajes se produce con \emph{Hausu} (Toho, 1977), un encargo en principio concebido como respuesta japonesa al éxito global de \emph{Jaws} (Spielberg, 1975), pero que termina convirtiéndose en un objeto fílmico radicalmente excéntrico dentro del catálogo de la compañía.\citep{House1977,HeartOfWeirdness2018} Ese mismo año dirige también \emph{Hitomi no naka no houmonsha} (\emph{The Visitor in the Eye}, 1977), consolidando su relación profesional con Toho.\citep{ObayashiSenses2021}

A partir de entonces, su actividad en el campo del largometraje se estabiliza y se diversifica: alterna cine juvenil protagonizado por \emph{idols}, relatos fantásticos, dramas literarios y proyectos de fuerte carga autobiográfica o histórica.

% =========================================================
\subsection{Obra cinematográfica: ciclos y etapas}
% =========================================================

La filmografía de Obayashi puede organizarse, con fines analíticos, en varios ciclos o etapas que no son estrictamente cronológicos, pero sí temático-estilísticos.

% ---------------------------------------------------------
\subsubsection{Cortometrajes experimentales (años 50 y 60)}
% ---------------------------------------------------------

Sus trabajos de 8\,mm y 16\,mm de los años cincuenta y sesenta constituyen uno de los núcleos fundacionales del cine experimental japonés de posguerra. Piezas como \emph{Complexe} (1964), \emph{Émotion} (1966) o \emph{Confession} (1968) exploran:

\begin{itemize}
	\item la autorreflexividad del dispositivo cinematográfico;
	\item la hibridación entre melodrama, humor absurdo y cita paródica del cine de terror occidental;
	\item el uso del montaje como ruptura del espacio-tiempo clásico, con saltos bruscos, repeticiones, velocidades alteradas y encadenados imposibles.
\end{itemize}

Estos cortos establecen muchas de las constantes formales de su obra posterior: predominio del truco sobre la ilusión, alegría consciente de la artificialidad y una relación casi táctil con el soporte fílmico.\citep{MesMidnightEye2009}

% ---------------------------------------------------------
\subsubsection{\emph{Hausu} (1977): síntesis de publicidad, horror y cultura pop}
% ---------------------------------------------------------

\emph{Hausu} constituye un punto de inflexión tanto en la carrera de Obayashi como en la historia del cine de horror japonés. El proyecto nace del deseo de Toho de producir un ``\emph{domestic Jaws}'', pero Obayashi plantea la película como traducción audiovisual de los miedos y fantasías de su hija Chigumi, de once años.\citep{HeartOfWeirdness2018,House1977} El resultado es una comedia de horror juvenil en la que siete colegialas visitan la mansión embrujada de una tía soltera y son devoradas ---literal y metafóricamente--- por la casa.

\emph{Hausu} opera simultáneamente en varios niveles:

\begin{enumerate}
	\item \textbf{Relectura del gótico y el \emph{kaidan}}: la casa embrujada y la tía espectral remiten a tradiciones del horror japonés, pero filtradas por el prisma pop y adolescente de los años setenta.
	\item \textbf{Explosión visual anti-realista}: efectos ópticos visibles, cromas imperfectos, animaciones dibujadas, sobreimpresiones y cambios abruptos de escala producen una suerte de \emph{le cinéma du WTF} en clave japonesa.\citep{CriterionObayashi2020}
	\item \textbf{Subtexto histórico y traumático}: la figura de la tía que espera eternamente a su prometido muerto en la guerra permite leer la película como alegoría de un Japón marcado por las ausencias del conflicto y por la domesticación televisiva del trauma.\citep{ObayashiSenses2021}
\end{enumerate}

La película fracasa en su estreno doméstico como producto estrictamente comercial, pero adquiere progresivamente estatus de \emph{cult film}, primero en circuitos cinéfilos japoneses y, desde la restauración y edición de Criterion, en la cinefilia internacional.

% ---------------------------------------------------------
\subsubsection{Cine juvenil, fantasía y el ``ciclo de Onomichi'' (años 80)}
% ---------------------------------------------------------

En los años ochenta, Obayashi deviene uno de los principales especialistas en cine juvenil y \emph{coming-of-age} dentro del sistema industrial japonés.\citep{ObayashiSenses2021,BeyondHouse2020} Entre sus títulos más relevantes destacan:

\begin{itemize}
	\item \emph{Tenkōsei} (\emph{I Are You, You Am Me}, 1982), donde dos adolescentes intercambian literalmente sus cuerpos;
	\item \emph{The Girl Who Leapt Through Time} (1983), adaptación de la novela de Yasutaka Tsutsui, que articula viaje temporal, melodrama y ciencia ficción ligera;
	\item \emph{Bound for the Fields, the Mountains, and the Seacoast} (1986), situada en la preguerra, que vincula infancia, violencia y militarización.
\end{itemize}

Estas obras, muchas de ellas rodadas en Onomichi o en espacios cercanos, suelen agruparse bajo la etiqueta de ``ciclo de Onomichi''. Suelen articular:

\begin{itemize}
	\item una reflexión sobre la memoria local frente a la homogeneización mediática;
	\item un interés por la experiencia adolescente como umbral entre inocencia y violencia histórica;
	\item una poética del espacio cotidiano transformado por la imaginación y el recuerdo.
\end{itemize}

% ---------------------------------------------------------
\subsubsection{Madurez, adaptaciones literarias y exploraciones extremas (1990--2000)}
% ---------------------------------------------------------

En los años noventa, la filmografía de Obayashi se diversifica aún más. Realiza adaptaciones literarias y biográficas que abordan la violencia, la sexualidad y la historia reciente con un tono más sombrío, aunque manteniendo su inclinación por la estilización formal. \emph{Sada} (1998), basada en el célebre caso de Sada Abe, es quizá el ejemplo más conocido: un filme que retoma un episodio emblemático del imaginario erótico-morboso japonés desde una perspectiva que combina distanciamiento formal, subjetividad exacerbada y comentario histórico.\citep{ObayashiSenses2021}

% ---------------------------------------------------------
\subsubsection{La trilogía antibélica contemporánea (2012--2017)}
% ---------------------------------------------------------

En la última etapa de su carrera, marcada por el diagnóstico de un cáncer avanzado, Obayashi consagra tres largometrajes monumentales a una reflexión sobre la guerra, la catástrofe y la responsabilidad intergeneracional:

\begin{itemize}
	\item \emph{Casting Blossoms to the Sky} (2012),
	\item \emph{Seven Weeks} (2014),
	\item \emph{Hanagatami} (2017).\citep{Hanagatami2017,WarTrilogyReview2021}
\end{itemize}

Con duraciones que rondan o superan las tres horas, estos filmes articulan múltiples líneas temporales, capas de memoria y una imaginería intensamente artificial (fondos digitales, composiciones imposibles, superposiciones reiteradas). A través de ellas, Obayashi busca, según sus propias palabras, enviar un mensaje ``de los adultos que vivieron el pasado a los niños del futuro''.\citep{LarkWarTrilogy}

Críticos posteriores han calificado esta trilogía como uno de los logros cinematográficos más singulares de la década de 2010.\citep{LATimesWarTrilogy2021}

% =========================================================
\subsection{Temas, motivos y procedimientos formales}
% =========================================================

% ---------------------------------------------------------
\subsubsection{Tiempo, memoria y nostalgia}
% ---------------------------------------------------------

El tiempo constituye quizá el motivo estructurante por excelencia en la obra de Obayashi. Sus películas vuelven una y otra vez sobre:

\begin{itemize}
	\item la infancia como espacio de memoria reconstruida;
	\item la adolescencia como umbral donde pasado y futuro se superponen;
	\item la guerra como herida temporal que desgarra la continuidad de la vida cotidiana.
\end{itemize}

No se trata de una memoria historicista, sino de una memoria afectiva, mediada por imágenes, relatos familiares, objetos y paisajes. El montaje fragmentario, la superposición de épocas y la inserción de fotografías o material de archivo dentro de la ficción articulan un régimen de tiempo que podríamos llamar \emph{estratigráfico}: varias capas de pasado, presente y futuro coexisten simultáneamente en la pantalla.\citep{ObayashiSenses2021}

% ---------------------------------------------------------
\subsubsection{El cuerpo y el espacio como superficies de inscripción}
% ---------------------------------------------------------

Otro rasgo central es el tratamiento del cuerpo y del espacio como superficies sobre las que se inscriben fuerzas históricas, afectivas y mediáticas. En \emph{Hausu}, el cuerpo de las colegialas se descompone en miembros, cabezas, torsos que flotan, desaparecen o son devorados por objetos domésticos; en las películas de Onomichi, el espacio urbano y marítimo aparece atravesado por capas de recuerdo e imaginación infantil; en la trilogía de guerra, los cuerpos envejecidos y enfermos conviven con la presencia espectral de la juventud perdida.\citep{MesMidnightEye2009,ObayashiSenses2021}

Obayashi rehúye sistemáticamente el realismo anatómico o espacial: sus cuerpos y sus paisajes son plásticos, manipulables, abiertos a la intervención del truco, de la animación y del montaje digital. En ello se aproxima tanto a tradiciones del cine de vanguardia como al imaginario del manga y del anime.

% ---------------------------------------------------------
\subsubsection{Montaje asociativo y artificialidad visible}
% ---------------------------------------------------------

Formalmente, su cine se caracteriza por una apuesta radical por la artificialidad visible:

\begin{itemize}
	\item encadenados y superposiciones que dejan ver las costuras del truco;
	\item fondos pintados, \emph{chroma keys} imperfectos, composiciones de estudio abiertamente teatrales;
	\item montaje que privilegia la asociación poética o emocional por encima de la continuidad espacial clásica.
\end{itemize}

Esta poética del truco asumido establece una posición estética y ética: el cine no es un medio para ``ocultar'' la mediación, sino un espacio para exponerla y jugar con ella, invitando al espectador a participar activamente en la construcción del sentido.

% =========================================================
\subsection{Obayashi y las vanguardias}
% =========================================================

La obra de Obayashi dialoga con varias corrientes vanguardistas, pero mantiene una autonomía marcada.

\begin{itemize}
	\item En relación con la \textbf{Art Theatre Guild} (ATG) y la \emph{Japanese New Wave}, comparte el interés por el anti-realismo, la crítica social y la experimentación formal, pero se distancia por su fuerte impronta pop, su humor lúdico y su preferencia por protagonistas adolescentes.
	\item Frente al \textbf{cine de vanguardia europeo} (Godard, la escuela estructural), su cine es menos ensayístico y más afectivo, anclado en historias concretas y en espacios locales.
	\item En comparación con el \textbf{New Hollywood} y otros cines juveniles internacionales de los setenta, Obayashi adopta la figura de la juventud como sujeto central, pero la sitúa en relación directa con la memoria de la guerra y con la especificidad de la historia japonesa.
\end{itemize}

Esta posición liminar ---entre industria y vanguardia, entre Japón y el imaginario global, entre publicidad y cine de autor--- explica tanto la dificultad inicial para canonizar su obra como la intensidad de su revalorización contemporánea.\citep{BeyondHouse2020}

% =========================================================
\subsection{Recepción crítica, canonización y legado}
% =========================================================

Durante buena parte de su carrera, Obayashi fue percibido por la crítica japonesa como un director excéntrico, asociado al cine juvenil, a la publicidad y a productos genéricamente difíciles de clasificar. La plena canonización de su obra es relativamente tardía y está vinculada, en gran medida, a dos procesos:

\begin{enumerate}
	\item la circulación internacional de \emph{Hausu} en festivales, ciclos de culto y, finalmente, en la edición restaurada de la Criterion Collection;
	\item la recepción crítica entusiasta de su trilogía antibélica en la década de 2010, que permitió releer retrospectivamente toda su filmografía a la luz de una preocupación constante por la memoria y la guerra.\citep{BeyondHouse2020,WarTrilogyReview2021,LWLiesWarTrilogy}
\end{enumerate}

Autores como Tom Mes, Jasper Sharp, Chris Fujiwara o Hal Young han contribuido a situarlo como figura clave del cine japonés moderno, subrayando tanto la coherencia de sus motivos como la singularidad de su lenguaje visual.\citep{MesMidnightEye2009,ObayashiSenses2021} Al mismo tiempo, diversos cineastas contemporáneos ---entre ellos Shunji Iwai, Sion Sono o Takashi Miike--- han reconocido la influencia de su libertad formal y su mezcla de ternura, violencia y humor.\citep{MidnightEyeGuide2004}

En la actualidad, la obra de Nobuhiko Obayashi puede leerse como un verdadero archivo sensible de la posguerra japonesa: un cine que, desde la infancia y la juventud, interroga de manera obstinada el legado de la guerra, la fascinación por la tecnología de la imagen y la capacidad del cine para reinventar, una y otra vez, las formas de recordar y de imaginar el futuro.

