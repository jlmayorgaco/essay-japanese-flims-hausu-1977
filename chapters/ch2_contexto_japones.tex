% =========================================================
\section{Contexto cinematográfico japonés (1965--1977)}
% =========================================================

El periodo comprendido entre mediados de la década de 1960 y 1970 constituye uno de los momentos más convulsos y transformadores de la historia del cine japonés. Las mutaciones en los modos de producción, la irrupción de nuevas sensibilidades juveniles, el colapso del sistema clásico de estudios y la emergencia de circuitos alternativos dieron lugar a un ecosistema fílmico heterogéneo, contradictorio y especialmente fértil para propuestas estéticas radicales como \textit{Hausu} (Obayashi, 1977). En esta sección se contextualizan cuatro vectores históricos esenciales para comprender la génesis industrial, cultural y formal de la película.

% ---------------------------------------------------------
\subsection{Crisis del sistema de estudios}
% ---------------------------------------------------------

A partir de 1965, la industria cinematográfica japonesa entró en un periodo de contracción acelerada cuya magnitud puede observarse en los datos de asistencia y exhibición. En 1960, las salas japonesas registraron aproximadamente 1{,}130 millones de espectadores; para 1965 la cifra había caído a 702 millones y, en 1970, descendió aún más hasta 356 millones.\footnote{Datos de la Motion Picture Producers Association of Japan (EIREN), estadísticas históricas de asistencia y número de salas (1960--1975).} La reducción acumulada entre 1960 y 1975 supera el 80\%, como se observa en la Tabla~\ref{tab:asistencia-salas} y en la Figura~\ref{fig:asistencia-salas-tv-japon1955-1990}, acompañada por un cierre masivo de salas: de 7{,}457 cines en operación en 1960 se pasó a 3{,}227 en 1970 y apenas 1{,}788 en 1975.

Toho, Shochiku y Toei ---los tres pilares del modelo clásico del cine japonés--- respondieron al colapso industrial de mediados de los sesenta mediante una serie de estrategias defensivas ampliamente documentadas en la literatura. Como señala Wada-Marciano, durante esta década ``the Japanese major film studios have barely managed to survive domestically during a long-term economic decline that began in the 1960s'' \citep{WadaMarciano2009}. La contracción de ingresos y la pérdida sostenida de espectadores obligaron a los estudios a aplicar políticas de reducción de costos, externalización parcial de producciones y reorganización de unidades internas, en línea con lo descrito por los estudios históricos sobre el sistema de estudios japonés \citep{JFDB2024}.

La magnitud del ajuste industrial puede apreciarse en los datos de producción: el número de largometrajes realizados por los estudios mayores alcanzó un máximo de 496 títulos en 1960, pero descendió hasta apenas 254 en 1969, evidenciando una caída de casi el 50\% en menos de una década \citep{MasumuraStudy}. Shochiku, en particular, experimentó graves dificultades financieras; como recoge su propio registro histórico, ``the studio encountered more financial difficulties in the 1960s'' \citep{BritannicaShochiku}. Este deterioro estructural condujo a un viraje hacia modelos de producción más baratos, rápidos y adaptables.

Por ejemplo, algunos datos concretos ilustran la magnitud de los costes de producción en la industria mayor japonesa: la película \emph{King Kong vs. Godzilla} (Toho, 1962) alcanzó un presupuesto estimado de **US \$432 000**. \citep{WikipediaKingKongGodzilla1962} A su vez, se ha estimado que hacia mediados de los años sesenta el coste medio de una película japonesa se situaba entre los **US \$1 y US \$2 millones**. \citep{JSTORGuestEditorPreface} Dado el extraordinario volumen de producciones y la caída de ingresos en ese periodo, estas cifras permiten inferir una presión innegable para reducir costes medios e incrementar la proporción de producción de bajo presupuesto dentro de estudios como Toho, Shochiku y Toei.

En paralelo, los estudios comenzaron a experimentar con híbridos genéricos diseñados para atraer a públicos adolescentes en un mercado cada vez más fragmentado. Toei, por ejemplo, impulsó a partir de 1970 la línea de violencia erótica conocida como \emph{Pinky Violence}, concebida explícitamente como respuesta a la caída de asistencia en salas y al auge de la televisión. Este ciclo integró elementos de explotación, acción, delincuencia juvenil y melodrama sensacionalista, dando lugar a una serie de filmes protagonizados por mujeres jóvenes en situaciones extremas.

Entre los títulos más representativos se encuentran \emph{Sex and Fury} (Ishii, 1973), que combina artes marciales y erotismo estilizado; \emph{Female Prisoner 701: Scorpion} (Itō, 1972), cuyo éxito dio origen a múltiples secuelas y consolidó la figura de la mujer vengadora; \emph{Terrifying Girls’ High School: Lynch Law Classroom} (Ishii, 1973), que mezcla sátira escolar con violencia gráfica; y \emph{Delinquent Girl Boss: Blossoming Night Dreams} (Ozawa, 1970), centrada en pandillas femeninas juveniles. Estos filmes, rápidos, baratos y formalmente flexibles, se convirtieron en dispositivos estratégicos para sobrevivir en un ecosistema dominado por la televisión y atravesado por la contracción del \textit{studio system} \citep{Weisser1998, Sharp2011, PinkFilm}.


En conjunto, estos procesos explican el surgimiento de un entorno de producción caracterizado por la precariedad pero también por la permeabilidad estética. La tolerancia hacia propuestas formales experimentales aumentó notablemente, abriendo espacio para cineastas procedentes de la publicidad, la televisión o el cine independiente. En este clima híbrido y en descomposición se inserta \textit{Hausu} (Obayashi, 1977), cuya estética fragmentaria y estructura no convencional son inseparables de la crisis del sistema cinematográfico japonés en su fase final \citep{Desser1988, Richie1990}.


Este clima de precariedad e improvisación abrió grietas donde cineastas jóvenes o marginales encontraron oportunidades. \textit{Hausu} emerge directamente de esta fase terminal: una película concebida dentro de Toho, pero producida bajo una lógica híbrida, publicitaria y fragmentaria que refleja la descomposición del sistema clásico.

%

\begin{figure}[htbp]
\centering
\begin{tikzpicture}
  \begin{axis}[
    width=0.95\textwidth,
    height=0.60\textwidth,
    xmin=1955, xmax=1990,
    ymin=0, ymax=1200,
    axis lines=left,
    xlabel={Año},
    ylabel={Valor (escala común 0--1200)},
    xtick={1955,1960,1965,1970,1975,1980,1985,1990},
    ymajorgrids=true,
    grid style={dashed, jpGray!50},
    ticklabel style={font=\small},
    label style={font=\small},
    legend style={
      at={(0.02,0.97)},
      anchor=north west,
      draw=none,
      fill=none,
      font=\small
    },
  ]

    % --- Serie 1: Entradas totales (millones) ---
    \addplot[
      thick,
      jpRed,
      mark=*,
      mark options={jpRed,scale=1.1}
    ]
    coordinates {
      (1955,923.6)
      (1960,1087.2)
      (1965,448.2)
      (1970,337.3)
      (1975,304.8)
      (1980,330.3)
      (1985,328.6)
      (1990,317.9)
    };
    \addlegendentry{Entradas (millones)}

    % --- Serie 2: Salas de cine (escaladas /10) ---
    \addplot[
      thick,
      hausuBlue,
      dashed,
      mark=square*,
      mark options={hausuBlue,scale=1.0}
    ]
    coordinates {
      (1955,518.4)
      (1960,745.7)
      (1965,464.9)
      (1970,324.6)
      (1975,244.3)
      (1980,236.4)
      (1985,213.7)
      (1990,183.6)
    };
    \addlegendentry{Salas de cine ($\times 10$)}

    % --- Serie 3: Hogares con TV (% × 12 para escalar a 0–1200) ---
    \addplot[
      thick,
      jpGreenTV,
      dotted,
      mark=triangle*,
      mark options={jpGreenTV,scale=1.0}
    ]
    coordinates {
      (1955,1*12)
      (1956,12*12)
      (1957,23*12)
      (1958,34*12)
      (1959,45*12)
      (1960,55*12)
      (1961,64*12)
      (1962,73*12)
      (1963,82*12)
      (1964,88*12)
      (1965,94*12)
      (1966,95*12)
      (1967,96*12)
      (1968,97*12)
      (1969,98*12)
      (1970,98*12)
      (1971,98.4*12)
      (1972,98.8*12)
      (1973,99*12)
      (1974,99*12)
      (1975,99*12)
      (1976,99*12)
      (1977,99*12)
      (1978,99*12)
      (1979,99*12)
      (1980,99*12)
      (1981,99*12)
      (1982,99*12)
      (1983,99*12)
      (1984,99*12)
      (1985,99*12)
      (1986,99*12)
      (1987,99*12)
      (1988,99*12)
      (1989,99*12)
      (1990,99*12)
    };
    \addlegendentry{Hogares con TV (\% × 12)}

  \end{axis}
\end{tikzpicture}

\caption{Evolución de la asistencia al cine, número de salas y adopción de televisión en Japón (1955--1990). Todas las series comparten la misma escala vertical (0--1200).}
\label{fig:asistencia-salas-tv-japon1955-1990}
\end{figure}
