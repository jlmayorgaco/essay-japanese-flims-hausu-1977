% =========================================================
\section{Influencia del Noh}
\label{ch:noh}
% =========================================================
El teatro \textit{noh} ocupa un lugar central en la imaginación estética japonesa y, al mismo tiempo, en muchas de las formulaciones críticas sobre el cine de terror japonés y su especificidad frente al modelo hollywoodense.\footnote{Sobre la persistencia del \textit{noh} como matriz cultural en el audiovisual japonés contemporáneo, véanse \cite{Brazell2006, Leiter2002}.} En el caso de \textit{Hausu}, la relación con el \textit{noh} no se establece a nivel de cita directa o de adaptación textual, sino a través de una serie de resonancias formales y tematizaciones —el tiempo suspendido, la casa embrujada como espacio liminar, la figura del espíritu femenino vengativo— que conectan el film de Obayashi con una genealogía más amplia de lo fantasmático en la cultura japonesa.

En esta sección se exploran tres dimensiones de esa influencia: (1) la configuración del espacio y el tiempo como ámbitos liminares; (2) la estilización del cuerpo, el gesto y el rostro; y (3) la articulación de lo femenino y lo espectral en clave ritual.

% ---------------------------------------------------------
\subsection{Espacio liminar y tiempo suspendido}
% ---------------------------------------------------------

El \textit{noh} organiza su dramaturgia en torno a espacios liminares: puentes, templos, santuarios, cruces de caminos donde el mundo de los vivos y el de los muertos se superponen. La escenografía mínima y el uso del \textit{hashigakari} (el puente por el que entra el actor principal) construyen un espacio que es menos un lugar real que un umbral donde se materializan recuerdos, resentimientos y apariciones.\citep{Brazell2006}

\textit{Hausu} retoma esta lógica liminar al situar la mayor parte de su acción en una casa de campo que funciona simultáneamente como espacio familiar, tumba y trampa sobrenatural. El viaje de las colegialas desde la ciudad hasta la casa de la tía puede leerse como un tránsito a través de un \enquote{puente} simbólico: el tren, el paisaje pintado, las transiciones ópticas y los fondos artificiales subrayan que el film abandona progresivamente el registro del realismo para entrar en un ámbito regido por otras leyes temporales. La casa, como en muchos dramas de \textit{noh}, no es un simple decorado sino un cuerpo que recuerda, devora y reconfigura a quienes la habitan.

La temporalidad del film también se aproxima a la del \textit{noh}. Aunque la narración de \textit{Hausu} parece avanzar linealmente, las rupturas de continuidad, los flashbacks estilizados (la historia del compromiso truncado de la tía) y las repeticiones de ciertos motivos visuales (la luna, el pozo, el vestido de novia) producen la sensación de un tiempo plegado sobre sí mismo, más cercano al \enquote{tiempo del recuerdo} (\textit{tsuioku}) del \textit{noh} que al tiempo causal del melodrama clásico. El clímax, en el que la casa entera se inunda de sangre y las protagonistas son absorbidas por el espacio, condensa esta lógica de tiempo suspendido: no hay resolución ni retorno a la normalidad, sino una especie de eternización del trauma.

% ---------------------------------------------------------
\subsection{Máscara, rostro y estilización del cuerpo}
% ---------------------------------------------------------

Uno de los elementos más distintivos del \textit{noh} es el uso de máscaras que, lejos de bloquear la expresión, funcionan como superficies altamente codificadas sobre las que pequeños cambios de inclinación o de iluminación generan variaciones dramáticas de emoción.\citep{Leiter2002} La actuación corporal —el \textit{kata}, o patrón de movimiento— y el ritmo vocal sustituyen al psicologismo por una expresividad ritualizada y frontal.

En \textit{Hausu}, Obayashi reinterpreta esta lógica de la máscara a través del dispositivo cinematográfico. Los rostros de las colegialas son constantemente intervenidos por la puesta en escena: congelados en primeros planos de sonrisa publicitaria, recortados por marcos dentro del encuadre, superpuestos con efectos ópticos o deformados por la animación. El célebre momento en que la cabeza decapitada de Kung Fu flota y muerde el trasero de una de sus compañeras no solo funciona como gag gore, sino como inversión grotesca de la máscara \textit{noh}: el rostro se separa del cuerpo, conserva una expresión fija y, al mismo tiempo, adquiere movimiento autónomo.

Algo similar ocurre con el cuerpo. Las poses exageradas, los saltos congelados y la coreografía de los ataques de la casa (el piano que devora a Melody, el reloj que atrapa a Gorgeous) convierten el gesto en figura casi coreográfica, más cercana al \textit{kata} teatral que a la actuación naturalista. En lugar de acceder al interior psicológico de las personajes, el espectador recibe una serie de signos corporales saturados que recuerdan tanto al \textit{noh} como a la publicidad televisiva: cuerpos tratados como signos gráficos, no como individuos profundos.

% ---------------------------------------------------------
\subsection{Lo femenino espectral: de la dama del Noh a la tía de \textit{Hausu}}
% ---------------------------------------------------------

Numerosas piezas de \textit{noh} giran en torno a figuras femeninas espectrales: espíritus de esposas abandonadas, amantes traicionadas o damas de corte consumidas por los celos, cuya aparición ritualizada sirve para reescenificar un trauma histórico o afectivo. Obras como \emph{Aoi no Ue}, \emph{Dōjōji} o \emph{Kinuta} articulan un imaginario donde lo femenino se asocia a la persistencia del rencor (\textit{urami}) y a una forma de violencia que es simultáneamente íntima y cósmica.\citep{Brazell2006, Leiter2002}

La tía de \textit{Hausu} puede leerse en continuidad con esta tradición. Su biografía —una mujer que pierde a su prometido en la guerra, permanece fiel a un vínculo imposible y termina devorando a las jóvenes que visitan su casa— condensa varios motivos del \textit{noh}: el amor no resuelto, la espera interminable, la transformación en espíritu vengativo. Obayashi traduce esta figura a un registro \textit{pop}: en lugar de aparecer con máscara y kimono, lo hace a través de trucajes, transparencias y cambios abruptos de iluminación que la convierten en una presencia inestable, a veces cómica y a veces terrorífica.

El propio diseño de la casa como extensión del cuerpo de la tía —las puertas que se cierran solas, los objetos que atacan, la sangre que inunda las estancias— recuerda a la manera en que, en el \textit{noh}, el espacio escénico se vuelve emanación del estado anímico del espíritu. La fusión final entre Gorgeous y la tía, en la que la sobrina adopta su kimono y su peinado y recibe a su amiga en un plano de calma sobrenatural, funciona como una suerte de \enquote{poscesión hereditaria}: el rencor femenino se transmite y se reactualiza, de forma análoga a como las obras de \textit{noh} reponen una y otra vez las mismas figuras espectrales.

% ---------------------------------------------------------
\subsection{Ritual, repetición y publicidad}
% ---------------------------------------------------------

Si el \textit{noh} se ha descrito a menudo como un teatro de la repetición ritual —el mismo repertorio, los mismos patrones de movimiento, variaciones mínimas en cada función—, la publicidad televisiva se basa en otra forma de repetición: la circulación insistente de eslóganes, jingles e imágenes diseñadas para fijarse en la memoria.\citep{Ivy1995} \textit{Hausu} se sitúa en la intersección de ambos regímenes: toma motivos del imaginario ritual (la casa como santuario, el recuerdo de los muertos, la ofrenda de los cuerpos de las jóvenes) y los recombina siguiendo una lógica de clip publicitario, con cortes abruptos, sobreimpresiones y un uso abiertamente lúdico de los efectos especiales.

En este sentido, la influencia del \textit{noh} en \textit{Hausu} no debe entenderse como una transferencia directa de formas tradicionales a un film de terror, sino como la reactivación, en clave \textit{pop} y televisiva, de ciertas estructuras profundas: el espacio liminar, el tiempo suspendido, la figura del espíritu femenino, la repetición ritualizada del trauma. Obayashi convierte esos elementos en materia de juego visual, pero sin disolver su dimensión inquietante. La película funciona, así, como un \enquote{\textit{noh} eléctrico}: un rito espectral que ha pasado por la lógica de la publicidad y la cultura juvenil de los setenta, pero que sigue organizando la experiencia del espectador en torno a la aparición y la persistencia de lo que no puede ser completamente exorcizado.

