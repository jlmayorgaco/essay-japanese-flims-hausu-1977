
\begin{figure}[htbp]
	\centering
	\begin{tikzpicture}
		\begin{axis}[
			width=0.95\textwidth,
			height=0.60\textwidth,
			xmin=1955, xmax=1990,
			ymin=0, ymax=1200,
			axis lines=left,
			xlabel={Año},
			ylabel={Valor (escala normalizada)},
			xtick={1955,1960,1965,1970,1975,1980,1985,1990},
			ymajorgrids=true,
			% Usamos el color oscuro nuevo con transparencia
			grid style={dashed, hausuDark!20}, 
			ticklabel style={font=\small\sffamily},
			label style={font=\small\sffamily\bfseries},
			legend style={
				at={(0.02,0.97)},
				anchor=north west,
				draw=hausuDark, % Borde negro pop
				fill=white,
				font=\small\sffamily,
				drop shadow % Sombra estilo cómic
			},
			]
			
			% --- Serie 1: Entradas totales (ROJO SANGRE) ---
			\addplot[
			very thick, % Más grueso para estilo Pop
			hausuBlood,
			mark=*,
			mark options={hausuBlood,scale=1.2}
			]
			coordinates {
				(1955,923.6) (1960,1087.2) (1965,448.2) (1970,337.3)
				(1975,304.8) (1980,330.3) (1985,328.6) (1990,317.9)
			};
			\addlegendentry{Entradas (millones)}
			
			% --- Serie 2: Salas de cine (TURQUESA/AZUL) ---
			\addplot[
			very thick,
			hausuTeal, % Color nuevo
			dashed,
			mark=square*,
			mark options={hausuTeal,scale=1.0}
			]
			coordinates {
				(1955,518.4) (1960,745.7) (1965,464.9) (1970,324.6)
				(1975,244.3) (1980,236.4) (1985,213.7) (1990,183.6)
			};
			\addlegendentry{Salas de cine ($\times 10$)}
			
			% --- Serie 3: TV (VERDE NEÓN) ---
			\addplot[
			very thick,
			hausuGreen, % El verde de los ojos del gato
			dotted,
			mark=triangle*,
			mark options={hausuGreen,scale=1.2, draw=black}
			]
			coordinates {
				(1955,12) (1960,660) (1965,1128) (1970,1176)
				(1975,1188) (1980,1188) (1985,1188) (1990,1188)
			};
			\addlegendentry{Hogares con TV (\% est.)}
			
		\end{axis}
	\end{tikzpicture}
	
	\caption{Colapso de la audiencia y auge de la TV (1955--1990).}
	\label{fig:asistencia-salas-tv-japon1955-1990}
\end{figure}