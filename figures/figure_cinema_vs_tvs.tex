\begin{figure}[htbp]
  \centering
  \begin{tikzpicture}

    % ---------- EJE IZQUIERDO: ENTRADAS + SALAS ----------
    \begin{axis}[
      width=0.95\textwidth,
      height=0.60\textwidth,
      xmin=1955, xmax=1990,
      ymin=0, ymax=1200,
      axis y line*=left,
      axis x line*=bottom,
      xlabel={Año},
      ylabel={Entradas (millones) / Salas ($\times 10$)},
      xtick={1955,1960,1965,1970,1975,1980,1985,1990},
      ymajorgrids=true,
      grid style={dashed, hausuDark!20}, 
      ticklabel style={font=\small\sffamily},
      label style={font=\small\sffamily\bfseries},
      legend style={
        at={(0.02,0.97)},
        anchor=north west,
        draw=hausuDark,
        fill=white,
        font=\small\sffamily,
        drop shadow
      },
    ]

      % --- Serie 1: Entradas totales (ROJO SANGRE) ---
      \addplot[
        very thick,
        hausuBlood,
        mark=*,
        mark options={hausuBlood,scale=1.2}
      ]
      coordinates {
        (1955,923.6) (1960,1087.2) (1965,448.2) (1970,337.3)
        (1975,304.8) (1980,330.3) (1985,328.6) (1990,317.9)
      };
      \addlegendentry{Entradas (millones)}

      % --- Serie 2: Salas de cine (TURQUESA/AZUL) ---
      \addplot[
        very thick,
        hausuTeal,
        dashed,
        mark=square*,
        mark options={hausuTeal,scale=1.0}
      ]
      coordinates {
        (1955,518.4) (1960,745.7) (1965,464.9) (1970,324.6)
        (1975,244.3) (1980,236.4) (1985,213.7) (1990,183.6)
      };
      \addlegendentry{Salas de cine ($\times 10$)}

    \end{axis}

    % ---------- EJE DERECHO: HOGARES CON TV (%) ----------
    \begin{axis}[
      width=0.95\textwidth,
      height=0.60\textwidth,
      xmin=1955, xmax=1990,
      ymin=0, ymax=100,
      axis y line*=right,
      axis x line=none,
      ylabel={Hogares con TV (\%)},
      ytick={0,20,40,60,80,100},
      yticklabel style={font=\small\sffamily},
      label style={font=\small\sffamily\bfseries},
      legend style={
        at={(0.98,0.03)},
        anchor=south east,
        draw=hausuDark,
        fill=white,
        font=\small\sffamily,
        drop shadow
      },
    ]

      % Datos aproximados en % (coherentes con la literatura)
      \addplot[
        very thick,
        hausuGreen,
        dotted,
        mark=triangle*,
        mark options={hausuGreen,scale=1.2, draw=black}
      ]
      coordinates {
        (1955,  1)  % TV incipiente
        (1960, 55)  % ~mitad de hogares
        (1965, 90)  % difusión casi completa en urbano
        (1970, 94)  % >90% hogares
        (1975, 98)
        (1980, 99)
        (1985, 99)
        (1990, 99)
      };
      \addlegendentry{Hogares con TV (\%)}

    \end{axis}

  \end{tikzpicture}

  \caption{Colapso de la audiencia cinematográfica y auge de la televisión en Japón (1955--1990).}
  \label{fig:asistencia-salas-tv-japon1955-1990}
\end{figure}

Los datos de asistencia y número de salas para el periodo 1955–1990 se han reconstruido a partir de las series oficiales de la Motion Picture Producers Association of Japan (EIREN), sintetizadas en \citep{EirenStats, Coates2020}.

Los valores aproximados de penetración televisiva (\%) representan una curva estilizada
a partir de diversos estudios sobre la difusión de la televisión en Japón, que sitúan
la presencia de aparatos en torno al 50--55\% de los hogares en 1960, por encima del
90\% en 1970 y cerca de la saturación (98--99\%) a finales de los años setenta y
ochenta \citep{Andrews2016, Yoshimi2010, StandardLivingJapan, TVPenetrationGraph}.
